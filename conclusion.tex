\chapter{Conclusion and Future Work}

The main contributions are listed below.

\begin{itemize}
	\item \textbf{A continuous and discrete model for freeway onramp metering and optimal control applications.} In Chapter~\ref{chapter:freeway-network-model}, we covered preliminary networked conservation law theory including the PDE formulation, Riemann solvers and Godunov discretization. From these tools, we derive a new model for linear freeway stretches, where the mainline flow is modeled as networked horizontal queues of vehicle density, while the onramps are modeled as ODE's of vertical queues of vehicle counts. The ODE allows for strong boundary conditions to be guaranteed at the onramp boundaries, and thus guaranteeing all applied boundary flow enters the system. We discuss the suitability of such a model to optimal control applications, where strong boundary conditions are desirable.
	\item \textbf{Adjoint-based finite horizon optimal control framework for networked conservation laws.} In Chapters~\ref{sec:discrete-adjoint-derivation} and~\ref{sec:adjoint-based-model}, we give an overview of the discrete adjoint method and its general applicability to optimal control problems. We derive a specific discrete adjoint formulation for networked conservation laws generalized about the specific Riemann solver being used in the application. Such a formulation allows for a study of the sparsity structure of such networked systems. We show that such a sparsity structure permits one to compute gradients of optimal control objectives on such systems with complexity linear in the size of the network and linear in the time horizon. We demonstrated the effectiveness and robustness of an adjoint-based MPC ramp metering controller on a model of the I15 South freeway, where our method was able to improve upon standard practicioners' techniques.
	\item \textbf{Distributed optimization over subsystems with shared state.} In Chapter~\ref{sec:admm-intro}, we derived a decentralized and asynchronous control algorithm over sub-systems which share not only ``free'' control variables, but ``dependent'' state variables. We first present a general treatment of the problem and solution method following the A-ADMM algorithm~\cite{Wei2013On}. Then we show how networked conservation laws, such as freeway networks, fit the assumptions of the presented problem, and how its specific sparsity pattern permits one to solve optimal control problems in parallel by splitting the network into subnetworks, with communication requirements that scale linearly with the network size. Furthermore, we show how the adjoint method can be applied to the subnetwork subproblems for systems with nonconvex dynamics. We then implemented a provably optimal, decentralized ramp metering and variable-speed-limit controller and demonstrated its improved running time with increasing subnetwork splits and its performance improvement over simpler decentralized approaches.
	\item \textbf{Security analysis of freeway control systems.} In Chapter~\ref{chapter:security}, we investigate the security and potential compromise points of freeway control systems, including by the physical and virtual aspects of control, sensing and communication. We distinguish between direct attacks, where the actuation is directly compromised, and indirect attacks, where sensing infrastructure is compromised in a manner which induces a desired outcome from the actuation. For coordinated ramp metering attacks, we construct a high-level framework, based on optimal control and multi-objective optimization, which enables an attacker to accomplish precise and intricate objectives using only metering lights as the control. A number of attack simulations are conducted on macroscopic freeway models which demonstrate the level of precisions possible from a coordinated ramp metering attack.
	\item \textbf{Rerouting strategies using mixed Lagrangian-Eulerian information.} In Chapter~\ref{chapter:le}, we present a framework for static route suggestions to a subset of users on flow networks occupied by non-compliant, greedy users. The framework only requires route-based, Lagrangian information from the compliant users, and flow counts on links from all users (Eulerian information). After applying a \emph{bounded-tolerance} model for the non-compliant drivers, we pose the optimal route suggestion problem as a convex optimization problem and give numerical examples applying the framework to both communication network dynamics and freeway dynamics.
\end{itemize}

During the course of conducting the above work, we identified a number of avenues for further research.

\begin{itemize}
	\item \textbf{Adjoint-based model calibration.} While using the discrete adjoint framework to conduct \emph{congestion-on-demand} attacks in Chapter~\ref{chapter:security}, it was identified that one could consider \emph{congestion-on-demand} objectives as model calibration, using onramp flow as the tuning model parameter. If one were to use more standard model parameters, such as the triangular fundamental diagram parameters and split ratios as the controllable, tuning parameters, then one can employ the discrete adjoint framework to minimize the model prediction error from some known sensor measurements by optimally adjusting the fundamental diagram parameters. A similar concept was presented in~\cite{Jacquet2005} for state estimation. As future work, one could study how such a calibration technique could be introduced into an MPC framework to allow for automatic, dynamic adjustment of freeway model parameters to account for unknowns such as weather or lane closures.
	\item \textbf{Sensitivity analysis of coordinated traffic subnetworks.} (Acknowledgment to Dr. Gabriel Gomes with PATH) The strength of the A-ADMM approach to distributed freeway control presented in Chapter~\ref{sec:admm-intro} comes from the transmission of not only boundary conditions to neighboring subnetworks, but objective value information via the Lagrangian dual variables. These dual variables also capture the \emph{sensitivity} of the objective to a particular dynamical constraint violation. Since constraint violation is equivalent to consensus enforcement in A-ADMM, by studying the dual variable values being transmitted between different subnetworks, one observes the impact of their communication on the objective value. Such analyses could inform traffic control systems designers on which subnetworks one should invest in enabling coordination, and which subnetworks do not require such investments.
\end{itemize}