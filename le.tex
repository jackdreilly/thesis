\chapter{Optimization-based Framework for Rerouting a Subset of Users with Mixed Lagrangian-Eulerian Demand}
\label{chapter:le}

\textbf{Note: } This work was part of a collaboration with Walid Krichene, Saurabh Amin, Kenza Skali-Lama and Alexandre Bayen. 

\section{Introduction}
\label{sec:le:introduction}


\subsection{Traffic assignment: selfish routing vs. social routing\label{sub:Traffic-assignment:-selfish}}

The problem of traffic assignment handles users' route and departure
time decisions and how individual behaviors impact the performance
of the underlying traffic network. If all user decide in a self-optimizing
manner, then the resulting network state is a \emph{user equilibrium~(e.g.
\cite{wardrop1952some})}. If every user acts in a manner that
is beneficial to societal goals, it is said to be a \emph{system}
or \emph{social optimum}. Socially optimal schemes are studied under
the assumption that a central agency controls \emph{all }the users,
while on the other extreme, user equilibrium is is a good model to
describe selfish behavior \emph{in the absence} of a central agency.
A complete characterization user equilibrium model requires complete
information of the origin-destination demands on the network. This
information is often too expensive to obtain. Specifically, origin-destination
information may only be available for a fraction of users because
collecting such information requires participation/consent of the
travelers and technological capability of the central agency. \cite{lo2002cell}
give a variational inequality approach to solving user equilibrium,
while \cite{papageogiou1990} presents an optimal control framework,
and both methods require full information of origin-destination demands
on the network.

These technologies can be broadly categorized into two categories.
First, there are recommendation systems, such as variable message
signs that suggest particular routes based on estimated travel times
or general dissemination of information to better inform users of
network conditions. Second, there are direct control systems that
restrict behavior of users via ramp metering or detours. \cite{gomes2008behavior}
discusses the effectiveness of ramp metering as a means of achieving
a social optimum. These direct control mechanisms are generally applied
at a specific point and time and do not distinguish between users
who have different routes or destinations. The effectiveness of such
active control schemes usually depends on complete origin-destination
demands. An exception is that boundary flow demands may be sufficient
for evacuation-type problems~(e.g. \cite{ziliaskopoulos2000linear}).


\subsection{Using mobile phones to control routes of individual users\label{sub:Using-cell-phones}}

With the emergence of GPS-enabled cell phones and their widespread
adoption in populated city areas, a third category of control has
become possible: one that communicates directly with users and permits
a central agency or a private entity to engage individual users to
shift their travel choices. Such a high granularity of control would
allow specific origin-destinations or routes to be targeted by the
control scheme and could even be customized to the route preferences
of the individual users.

Vehicle navigation services that collect, aggregate, and process information
from a large number of GPS-equipped mobile devices have become increasingly
popular. Such services include Waze, Google Maps, and other such mobile
applications. While these services are popular for their utility to
individual drivers, the service providers are also able to collect
information on behavior of the fraction of users that are equipped
with these devices. Once the data has been anonymized to protect the
privacy of individual users, the origin-destination information could
be interpreted as a subset of the total demand on a network. Additionally,
route guidance decisions could be made to benefit their user-base
as a whole, rather than on an individual level. \cite{Papadimitriou2001}
discusses the inherent inefficiencies of selfish routing versus the
social optimum.

Individually-applied control schemes have many advantages, but a limitation
is that the user-base of a particular vehicle navigation service would
only constitute a subset of the total users of the network. A significant
number of users of the network may not have access to or prefer not
to use a GPS-enabled device. Also, a complete understanding of the
origin-destination demands on a network by a single entity would still
be difficult or very expensive to obtain.


\subsection{Combining route-based demands with link-level flow information\label{sub:Scenario:-combining-route-based}}

\begin{figure}
\centering
\subfloat[]{%
\label{fig:Information-collection-methods}%
\includegraphics[width=.48\columnwidth]{previous-articles/le/le-figures/SystemArchitecture}%
}\hfill%
\subfloat[]{%
\label{fig:Communication-between-informatio}%
\includegraphics[width=.48\columnwidth]{previous-articles/le/le-figures/CommunicationDiagram}%
}
\protect\caption[Route guidance system architecture for using both route-based and
link-level flow information.]{Route guidance system architecture for using both route-based and
link-level flow information. \textbf{(a)} An illustration of Lagrangian
and Eulerian traffic information collection systems. \textbf{(1)}
Path-based information is collected via GPS-equipped vehicles, from
either onboard route-guidance systems or cell phones. \textbf{(2)}
Route-based information is sent back to a vehicle navigation service
that aggregates traffic information from many GPS-equipped users.
\textbf{(3)} Eulerian-based loop detectors collect flow counts and
send the information to a traffic management agency. \textbf{(b)}
An illustration of proposed interaction the traffic management agency,
vehicle navigation services, and network users. \textbf{(I)} Contracting
of the vehicle navigation services by the traffic management agency.
This may involve monetary compensation or tolling. \textbf{(II)} Anonymized
Lagrangian information (owned by the vehicle navigation agency) is
transferred to the traffic management agency. \textbf{(III)} The traffic
management service provides route guidance to the vehicle navigation
service to improve overall traffic conditions. \textbf{(IV)} The vehicle
navigation service provides individual network users with alternate
route suggestions, with potential incentivization. Users may be guided
to switch from their previously preferred (nominal) route.\label{fig:Lagrangian-Eulerian-route-guidan}}
\end{figure}


While collecting route information on individual users suffers from
limited penetration, existing traffic monitoring systems, such as
loop detectors or cameras, are able to capture all vehicle flows for
particular locations on networks. These stationary systems are often
monitored by public, traffic management agencies, that are interested
in the welfare of all users on the network. It is apparent that the
two methods for capturing traffic information are complementary: GPS-based
methods have limited penetration but more detailed origin-destination
information, while stationary sensors have full penetration of flow,
but cannot give route-level information on the demands. Figure \ref{fig:Information-collection-methods}
depicts the vehicle navigation service collecting aggregate GPS data
from their ``dark'' users on the network \textbf{(1)}, while the
traffic management agency collects flow count data, accounting for
\emph{all} users (``dark'' and ``light'') \textbf{(2)}, from the
loop detectors embedded in the road. Collectively, the route-based
flow data could be used as a source of \emph{re-routable }traffic
flow, while the link-based flow data could be used to better estimate
the expected travel times that all users will experience before and
after re-routing a subset of users. This chapter proposes a method
for using both information types to improve traffic conditions across
the network. For now, we motivate our work with a scenario in which
such a technique could be employed.

Figure \ref{fig:Communication-between-informatio} depicts the scenario
in which a traffic management agency partners with a vehicle navigation
service to take advantage of their different information sources.
Initially, a contractual phase may take place \textbf{(I)}, where
the public agency compensates the vehicle navigation service for access
to their data \textbf{(II)}. The information from the vehicle navigation
service would be aggregated and anonymized, in order to protect the
privacy of the individual users of the service. Then, the traffic
management agency would input the route-based demand data and the
stationary, link-based loop detector data into the algorithm we have
developed (Sections \ref{sec:le:Modeling-Partial-Participance} and \ref{sec:le:Accounting-for-Response}).
The algorithm outputs a new set of aggregated routing suggestions,
which are then sent to the vehicle navigation service \textbf{(III)}.
Finally, the service relays the routing suggestions to their users
\textbf{(IV)}. 

For this last point of communication between the agency and the user,
there are two notes. First, it is likely that a fraction of users
will be suggested routes with larger travel times than to which they
have become accustomed. In order to incentivize the user to accept
the route suggestion, the traffic management agency may require the
vehicle navigation service to compensate these users, enforceable
by the initial contractual agreement between the two organizations.
\cite{Merugu2009} describes an experiment which utilized incentives
to move commuters' departure times to less congested times. Alternatively,
one can consider that socially optimized routing policies may decrease
the travel time for all users \emph{on average}. Then, if all users
get assigned desirable routes some days and less desirable routes
other days (in order to reduce congestion on desirable routes), then
every user could expect to have an improved average travel time. Such
an argument could potentially remove the need for monetary compensation
or other types of incentivizes. The second note is that we have assumed,
given enough incentive from the vehicle navigation service, a user
will \emph{always} comply with the suggested route. We do not discuss
the method of incentives in this chapter, but note that the assumption
can be relaxed by limiting the amount of re-routable flow.
\begin{figure}[h]
\centering
\includegraphics[width=.5\columnwidth]{previous-articles/le/le-figures/MultiService}%
\caption[Scenario with multiple vehicle navigation services. Individual service's
data is aggregated by the central agency and not shared with other
agencies.]{Scenario with multiple vehicle navigation services. Individual service's
data is aggregated by the central agency and not shared with other
agencies. The routing strategies are calculated by the traffic management
agency, and the suggested routes are partitioned between the different
user groups.}
\label{fig:Scenario-with-multiple}
\end{figure}


Due to the decoupled nature of the system described in Figure \ref{fig:Lagrangian-Eulerian-route-guidan},
we can generalize the scenario to include multiple vehicle navigation
services (Figure~\ref{fig:Scenario-with-multiple}). Without sharing
information between services, more route-based flow information can
be used as input into the algorithm, thus providing more complete
information on the origin-destination preferences of the users and
collecting a larger pool of re-routable users, while maintaining the
privacy of the services which wish to provide socially optimal routes
to users.


\subsection{Accounting for untracked users' response\label{sub:Accounting-for-untracked}}

There are a number of reasons why a user would participate in the
socially optimized routing guidance program described above. As already
stated, they could be incentivized through monetary compensation.
They could also simply be altruistic, and willing to sacrifice personal
optimality for the greater good. What is unknown is the behavioral
aspects of those users of the network whose route cannot be tracked.
How can one predict the response of these untracked (which we refer
to as noncooperative later) network users to the routing schemes being
implemented for the tracked users?

A standard approach, described as a Stackelberg game~(e.g.~\cite{roughgarden2001stackelberg,Krichene2012a}),
assumes that the users outside the control of the central agency will
respond with a user equilibrium assignment. Since the origin-destination
demands of the untracked users is unknown, solving for a user equilibrium
is not possible.

In order to address this lack of information on preferences of untracked
users, we develop an alternative model of behavior. Related to the
concept of bounded rationality in \cite{Guo2011,Hu199751}, we assume
that the untracked users lack the full information of the state of
the network, and cannot make fully rational decisions on their optimal
route. Alternatively, the untracked users could possess some \emph{inertia}
towards switching routes, and will be content with their previously
chosen (nominal) routes, as long as the experienced travel time on
the route does not change ``too much''. This concept of inertia
can be practically motivated by considering that some users may appreciate
the scenic beauty of a particular suboptimal route, or others have
a favorite café along another route. Thus, in order to reasonably
assume that the untracked users will not switch their routes, the
routing suggestions provided by the algorithm are guaranteed to not
significantly deteriorate the quality of existing routes, beyond an
a priori specified bound.

A bounded rationality argument in the context of drivers' route selections
was made in \cite{Hu199751}, where drivers only seek utility gains
outside of a certain threshold. \cite{Guo2011} give some empirical
evidence of bounded rationality on road networks. Our model differs
from these because our model lacks origin-destination information
on the noncooperative users, and to make this distinction, we refer
to our model as \emph{bounded tolerance} model.


\subsection{Contributions and overview\label{sub:Contributions-and-overview}}

There is relatively little work done on how partial control schemes
can be practically implemented on flow networks. Additionally, inconsistent
estimations of traffic between GPS-based data and link-level data
can complicate the analysis of the problem. In this chapter, we present
a single methodology for accommodating both origin-destination based
and link-level flow information for a general, multi-origin, multi-destination,
static network (parameters are unchanging with time), while guaranteeing
that the two sources of data are consistent with mass balance across
junctions (Section~\ref{sec:le:Modeling-Partial-Participance}). Furthermore,
we present a behavioral model on the untracked users based on the
concept of bounded rationality (Section~\ref{sec:le:Accounting-for-Response}).
This bounded rationality model permits one to cope without origin-destination
demands for all users on the network, while still addressing the behavioral
aspects of self-routing users. 

As our main contribution, we demonstrate how the models presented
in this chapter lead to an elegant, optimization-based solution to
the socially optimal routing strategy problem (Section~\ref{sec:le:Convex-Optimization-Program}).
The optimization problem is proven to be convex for a specific instance
of horizontal queues that model highway traffic and extended to a
general class of vertical queues . As a corollary, we show that for
the discretized LWR network model, the social optimum can be solved
exactly for both the purely Eulerian flow and the purely Lagrangian
flow cases. 

The generality of our method is given by applying the framework to
a multiple-destination network with horizontal queues and investigating
how changes in the tolerance model impact the routing advice~(Section~\ref{sec:le:Numerical-Results}).
The chapter finishes with a conclusion and discussion of the practical
importance of the framework and models developed here-within (Section~\ref{sec:le:Conclusion}).


\section*{Nomenclature}
\label{sec:le:Nomenclature}

\begin{longtable}{ll}
\label{tab:le:nomenclature}
$\links$ & Set of links.\tabularnewline
$\origins\subset\links$ & Set of origins (sources).\tabularnewline
$\destinations\subset\links$ & Set of destinations (sinks).\tabularnewline
$\junctions$ & Set of junctions.\tabularnewline
$\routes$ & Set of routes.\tabularnewline
$\route\in\routes$ & Sequence of contiguous links $\left(\route_{1},\ldots,\route_{\left|\route\right|}\right):\route_{i}\in\links$\tabularnewline
$\inlinks{\junction}\subset\links$ & Set of incoming links for junction $\junction\in\junctions$.\tabularnewline
$\outlinks{\junction}\subset\links$ & Set of outgoing links for junction $\junction\in\junctions$.\tabularnewline
$\junctionroutes{\junction}\subset\routes$ & Set of routes passing through junction $\junction\in\junctions$.\tabularnewline
$\linkflow{\link},\prevlinkflow{\link}$ & Flow (resp. nominal flow) on link $\link\in\links$.\tabularnewline
$\routeflow{\route}$ & Total flow on route $\route\in\routes$\tabularnewline
$\comprouteflow{\route}$ & Cooperative flow on route $\route\in\routes$.\tabularnewline
$\density_{\link},\bar{\density}_{\link}$ & Density (resp. nominal density) on link $\link\in\links$\tabularnewline
$\veccomprouteflow$ & Assignment of cooperative flows across all routes $\in\routes$.\tabularnewline
$\prevnclinkflow{\link}$ & Nominal noncooperative flow on link $\link\in\links$.\tabularnewline
$\odflow{\origin}{\destination}$ & OD flow demand of cooperative (Lagrangian) users from origin $\origin\in\origins$
to destination $\destination\in\destinations$.\tabularnewline
$\latency_{\link},\origlatency_{\link}$ & Latency (resp. nominal latency) on link $\link\in\links$.\tabularnewline
$\alpha$ & Tolerance scale factor.\tabularnewline
 & \tabularnewline
\end{longtable}



\section{Modeling partial cooperation with Lagrangian-Eulerian demands}
\label{sec:le:Modeling-Partial-Participance}

\begin{wrapfigure}{r}{0.4\columnwidth}%
\begin{centering}
\includegraphics[width=0.4\columnwidth]{previous-articles/le/le-figures/data_flow_diagram}
\par\end{centering}

\protect\caption[Data-flow diagram for the partial compliance routing problem.]{Data-flow diagram\label{fig:data-diagram}}
\end{wrapfigure}%
We present the general setting of the routing problem considered,
as summarized in Figure~\ref{fig:data-diagram}. Consider a setting
in which a subset of users are equipped with GPS-enabled devices and
are connected to a central coordinator through a \emph{Routing interface}
(e.g. a mobile phone application). We refer to this subset as \emph{cooperative
}users. First, the cooperative users provide their desired routes
to the coordinator through the routing interface. This allows the
coordinator to have individual route information, i.e. \emph{Lagrangian
information} for the cooperative users. Second, the loop-detectors
(or other sensors capable of measuring aggregate link-level flows)
provide Eulerian information. We refer to the historical estimates
of Lagrangian and Eulerian information as the \emph{nominal} state
of the network.

Given the nominal Eulerian flow measurements for the entire network
and the nominal Lagrangian information for the equipped vehicles,
the central coordinator determines the optimal route assignment for
the equipped vehicles (Section~\ref{sub:Minimizing-Total-Travel}).
This optimization problem is represented by the \emph{optimal router}
block. Since only the cooperative users follow the optimal route assignments
provided by the central coordinator, we will refer to this problem
as a \emph{partial cooperation} problem.

The next step is an \emph{incentivization} step: given the target
optimal routes, and possibly additional constraints (such as a total
available budget) a second problem (not discussed in this chapter)
determines an incentive for each equipped vehicle and the corresponding
target route. The incentivization problem is outside of the scope
of the present chapter. More information on how to solve incentivization
and traffic demand management can be found in \cite{romain2012}.The
assigned routes and the corresponding incentives are then offered
to the equipped drivers, who can either accept or refuse the offer.
The subset of vehicles that do accept the offer (thus taking the route
assigned by the central coordinator) are called \emph{cooperating
vehicles}. In the present chapter, we focus our attention on the optimal
route assignment with information on mixed Lagrangian-Eulerian demands.

Considering the route optimization goals stated above, we give a declaration
of the problem statement to direct the model development of the proceeding
sections.
\begin{verse}
\textbf{Problem statement: }Find a mathematical framework for flow
networks which can encompass:
\begin{itemize}
\item Two different types of demand information: Lagrangian information,
which is specified by the route traversed by the flow, and Eulerian
information, which is specified by the flow-count across a link.
\item Socially optimal routing strategies which can encompass both information
types, given their limitations:

\begin{itemize}
\item Lagrangian information is only known for the cooperative flow, which
can be rerouted from its nominal route to improve network conditions.
\item Only Eulerian information is known for the noncooperative flow, which
is assumed to maintain its nominal state.
\end{itemize}
\end{itemize}
\end{verse}

\subsection{Network model\label{sub:Network-Model}}

Using standard network notation, the network model is defined by the
tuple, $\left(\links,\junctions\right)$, where $\links$ is the set
of links, and $\junctions$ is the set of junctions. A junction $\junction\in\junctions$
has a set of incoming links $\inlinks{\junction}\subseteq\links$
and outgoing links $\outlinks{\junction}\subseteq\links$. An origin
$\origin\in\origins\subseteq\links$ is a link with no upstream junction.
A destination $\destination\in\destinations\subseteq\links$ is a
link with no downstream junction. A route $\route=\left(\route_{1},\ldots,\route_{\left|\route\right|}\right)\in\routes$
is a set of adjacent links where $\route_{1}\in\origins$, $\route_{\left|\route\right|}\in\destinations$,
and $\forall i\in\left(1,\ldots,\left|\route\right|-1\right)$, $\exists\junction_{i}^{\route}:\route_{i}\in\inlinks{\junction_{i}^{\route}},\route_{i+1}\in\outlinks{\junction_{i}^{\route}}$.


\subsection{Cooperative demand vs. total demand}

Let the network in Section \ref{sub:Network-Model} contain flows
$\linkflow{\link}$ on every link $\link\in\links$. Furthermore,
we assume that the network is in steady state, i.e. all state on the
network is stationary with respect to time (e.g. flows). We further
differentiate two types of demands: Lagrangian and Eulerian.
\begin{itemize}
\item We assume that the cooperative users have provided their desired origin
and destination. Therefore, for every origin-destination pair $\left(\origin,\destination\right)\in\origins\times\destinations$,
there is a nominal flow demand $\odflow{\origin}{\destination}$ from
the cooperative users, where the bar notation refers to nominal state
values. Since this type of demand concerns the routes taken by the
flow, we describe this type of demand as \emph{Lagrangian }demand\emph{.}
\item For the noncooperative users, i.e., the users who do not (or choose
not to) interact with the routing interface, we do not assume knowledge
of Lagrangian demand. Thus, we assume that the only the aggregate
link-level flows are available via loop detectors. This aggregate
level information does not include OD and route information, and is
therefore defined as \emph{Eulerian} demand.
\end{itemize}
To recover the nominal Eulerian demand of the noncooperative vehicles,
we further assume that the nominally used routes of the cooperative
vehicles are known. For each link $\link\in\links$, we specify a
nominal total link flow $\prevlinkflow{\link}$, and for each route
$\route\in\routes$, we can specify a nominal route flow for cooperative
vehicles, $\prevcomprouteflow{\route}$. Then, the nominal noncooperative
Eulerian demand, $\prevnclinkflow{\link}$, is obtained for each link
$\link\in\links$ by subtracting cooperative flow from the total link
flow: 

\begin{equation}
\prevnclinkflow{\link}=\prevlinkflow{\link}-\sum_{\routesthroughlink}\prevcomprouteflow{\route}\label{eq:prevnclinkflow}
\end{equation}


For the remainder of the chapter we use the noncooperative link flows
($\prevnclinkflow{\link},\link\in\links$) as the input data for nominal
flow, but it is understood that this data is derived from the more
practically measurable \emph{total }nominal flow values ($\prevlinkflow{\link},\link\in\links$)
and the cooperative nominal route flows ($\prevcomprouteflow{\route},\route\in\routes$)
via Equation (\ref{eq:prevnclinkflow}). 

Since we have subtracted off the flow of nominal cooperative flow
to obtain the noncooperative flow, we study properties of the network
flow when the rerouted cooperative flow is added back into the network.
We introduce the decision variable: $\comprouteflow{\route}$, the
amount of cooperative flow assigned to route $\route\in\routes$.
To enforce that the entire flow across a link is accounted for and
same origin-destination demands of the cooperative users are satisfied,
we have the following constraints:

\begin{eqnarray}
\sum_{\routesthroughod}\comprouteflow{\route} & =\odflow{\origin}{\destination} & \forall\origin\in\origins,\destination\in\destinations\label{eq:od-flow}\\
\linkflow{\link} & =\sum\limits _{\routesthroughlink}\comprouteflow{\route}+\prevnclinkflow{\link} & \forall\link\in\links\label{eq:link-flow}
\end{eqnarray}


where $\odflow{\origin}{\destination}=\sum_{\routesthroughod}\prevcomprouteflow{\route}$
is cooperative flow between origin $\origin$ and destination $\destination$.

A requirement of the Eulerian flow is that noncooperative flow must
be conserved across junctions. If the flow across a link $\link\in\links$
is $\linkflow{\link}$, then the following must hold:

\begin{eqnarray}
\sum_{\link\in\inlinks{\junction}}\linkflow{\link} & =\sum\limits _{\link\in\outlinks{\junction}}\linkflow{\link} & \forall\junction\in\junctions\label{eq:mass-balance}
\end{eqnarray}


Since we partitioned flow on each link into two classes (cooperative
and noncooperative), flow conservation must hold across both classes
independently. We will see shortly that flow conservation across the
cooperative class will be guaranteed by the condition that all cooperative
flow must be assigned to a route. Then, for the noncooperative class
of nominal flow, we must always have the condition that flow conservation
holds across junctions. Since this is a condition on the input to
the problem we only state it here once and assume the condition for
the rest of the chapter.

\textbf{Model Consistency Condition: }For every junction $\junction\in\junctions$,
we assume: $\sum_{\link\in\inlinks{\junction}}\prevnclinkflow{\link}=\sum_{\link\in\outlinks{\junction}}\prevnclinkflow{\link}$.

Equations~(\ref{eq:od-flow})-(\ref{eq:mass-balance}) define a route-allocation
policy $\veccomprouteflow=\left\{ \comprouteflow{\route}:\route\in\routes\right\} $
for all cooperative users that satisfies all demand requirements.
There are three main requirements that we have from the set of constraints:
non-compliant (Eulerian) demand is satisfied, compliant (Lagrangian)
demand is satisfied, and mass balance across junctions is satisfied.
The first two are obvious from the above constraints, while the third
one needs proof.
\begin{prop}
For a feasible $\veccomprouteflow$ to the set of Equations (\ref{eq:od-flow})
and (\ref{eq:link-flow}), $\forall\junction\in\junctions$, $\sum_{\link\in\inlinks{\junction}}\linkflow{\link}=\sum_{\link\in\outlinks{\junction}}\linkflow{\link}$.\end{prop}
\begin{proof}
From the model consistency condition above, we only need to prove
the following statement: 
\[
\sum_{\link\in\inlinks{\junction}}\sum_{\routesthroughlink}\comprouteflow{\route}=\sum_{\link\in\outlinks{\junction}}\sum_{\routesthroughlink}\comprouteflow{\route}
\]
Let $\injunctionroutes{\junction}$ be the routes that pass through
links in the incoming links of junction $\junction$. Let $\outjunctionroutes{\junction}$
be the same for outgoing links. Then $\injunctionroutes{\junction}=\left\{ \route\in\routes:\inlinks{\junction}\cap\route\neq\emptyset\right\} $.
We also know that by the definition of a route, any route that passes
through an incoming link of a junction (not a source or sink) must
pass through an outgoing link, and therefore $\injunctionroutes{\junction}\subseteq\outjunctionroutes{\junction}$.
A similar argument can be made to show $\outjunctionroutes{\junction}\subseteq\injunctionroutes{\junction}$.
This shows that $\injunctionroutes{\junction}=\outjunctionroutes{\junction}$.
Then,

\[
\sum_{\link\in\inlinks{\junction}}\sum_{\routesthroughlink}\comprouteflow{\route}=\sum_{\injunctionroutes{\junction}}\comprouteflow{\route}=\sum_{\outjunctionroutes{\junction}}\comprouteflow{\route}=\sum_{\link\in\outlinks{\junction}}\sum_{\routesthroughlink}\comprouteflow{\route}
\]

\end{proof}

\subsection{Reducing total latency by rerouting cooperative users\label{sub:Minimizing-Total-Travel}}

We now formulate the problem of minimizing total latency (or equivalently,
total travel time) with route assignments of cooperative vehicles
as the decision variable. There are two classes of latency functions
studied in the literature: first~\cite{Nie2010c,roughgarden2001stackelberg,Papadimitriou2010},
where the link latency is assumed to be the function of flow in the
link, and second~\cite{Tong2000,lo2002cell,daganzo1994cell,lighthill1955kinematic,richards1956shock},
where density is assumed to affect link latencies. We generically
introduce latency as a value $\latency_{\link}$ associated with the
state and properties of a link $\link\in\links$ and discuss the different
models of latency as it pertains to different flow models in Section
\ref{sec:le:Convex-Optimization-Program}. We can therefore express the
total latency on a link as the flow times the latency, or $\linkflow{\link}\latency_{\link}$.

We can now express a general form of the Lagrangian-Eulerian flow,
route assignment problem in a standard optimization program formulation:

\begin{eqnarray}
\begin{array}{rcc}
\underset{\linkflow{\link}|\link\in\links,\comprouteflow{}|\route\in\routes}{\text{minimize}} &  & \sum_{\link\in\links}\linkflow{\link}\latency_{\link}\\
\text{subject to:}\\
 &  & \begin{aligned}\linkflow{\link} & = &  & \sum\nolimits _{\routesthroughlink}\comprouteflow{\route}+\prevnclinkflow{\link} &  & \forall\link\in\links\\
\sum_{\routesthroughod}\comprouteflow{\route} & = &  & \odflow{\origin}{\destination} &  & \forall\origin\in\origins,\destination\in\destinations
\end{aligned}
\end{array}\label{eq:le}
\end{eqnarray}


where the objective represents the total latency in the network, the
first constraint relates cooperative route flows and noncooperative
link flows to total link flows, and the last constraint states that
the compliant route flows must be partitioned in a way that satisfies
the nominal origin-destination demand.



\section{Accounting for response of noncooperative demand via bounded tolerance}
\label{sec:le:Accounting-for-Response}

Recall from our discussion in Section~\ref{sec:le:Modeling-Partial-Participance},
that due to imperfect perceptions of the travel times, the travelers
forming the non-cooperative demand may be assumed to be boundedly
rational, and may not change their nominal paths. This assumption
is especially valid when the travel times of non-cooperative users
do not significantly change when the cooperative users are routed
in a socially optimal manner. In contrast, if the cooperative vehicles
cause an excessive increase in latency on some routes for noncooperative
vehicles, then this assumption may not be realistic~(see \cite{aswani2011game,roughgarden2001stackelberg,krichenetac}).
For this reason, we have to enhance the model for rerouting cooperative
vehicles under stricter conditions.


\subsection{Bounded tolerance\label{sub:Motivating-limited-complacency}}

A traditional approach to predicting vehicle route choice comes from
the field of traffic assignment (e.g. \cite{wardrop1952some,lo2002cell,papageogiou1990})
and Nash equilibria in game theory (e.g. \cite{roughgarden2002bad,Papadimitriou2010}),
often described as \emph{user equilibrium} in the context of traffic
assignment and introduced in \cite{wardrop1952some}. The congestion
games literature considers Stackelberg games, which are used to analyze
how selfish users respond to a centrally-controlled subset of users
in a principled way (see \cite{Krichene2012a,roughgarden2001stackelberg}).
Our approach in this chapter is simpler. The reasonability of our approach
can be argued using a bounded tolerance assumption on the part of
noncooperative vehicles. We replace the assumption of stationarity
with a stronger assumption that stationarity is only achieved if all
routes on the network do not have a latency increase greater than
a certain amount, proportional to the nominal latency experienced
before rerouting the cooperative vehicles. Tolerance is assumed in
the sense that if latencies on a route do not noticeably increase,
then noncooperative vehicles do not seek better paths. However, the
tolerance to increase delay is still limited in the sense that as
latency increases on a route, noncooperative vehicles will eventually
switch. The term \emph{tolerance }is used to address the fact that
the bounded rationality models assumed in~\cite{Hu199751,Guo2011}
allow decisions to be made by the noncooperative users, while our
bounded tolerance model specifies how much perturbation of the nominal
is allowed, before the assumption the assumption is no longer valid
that noncooperative users do not change routes.


\subsection{Modeling bounded tolerance\label{sub:Modeling-limited-complacency}}

The discussion in Section \ref{sub:Motivating-limited-complacency}
dictates that one must have knowledge about a nominal network state
with which to compare final network state conditions. Therefore, we
introduce as input, the nominal latency $\origlatency_{\route}$ for
every route $\route\in\routes$. We then select as a model of bounded
tolerance the condition that the final latency on a route may not
be a factor $\left(1+\complacencyscale\right)$ greater than the nominal
latency $\origlatency_{\route}$, where $\alpha\in\mathbb{R}_{+}$
can be seen as the \emph{tolerance scale factor}. The latency on a
route can be computed by summing the latencies experienced on all
links in $\route=\left\{ \route_{1},\ldots,\route_{\left|\route\right|}\right\} $:

\[
\latency_{\route}=\sum_{\link\in\route}\latency_{\link}
\]


We can now express the bounded tolerance condition constraints:

\begin{eqnarray}
\sum_{\link\in\route}\latency_{\link} & \le\left(1+\complacencyscale\right)\origlatency_{\route} & \forall\route\in\routes\label{eq:limcomp-nonconvex}
\end{eqnarray}


Adding this constraint to Problem~\ref{eq:le} completes the \emph{partial
cooperation, bounded tolerance} problem. Section~\ref{sec:le:Convex-Optimization-Program}
describes how this problem applies to different flow models and latency
models. We now introduce another tolerance model and describe the
class of problems to which it can be applied.


\subsection{Comparative tolerance\label{sub:Comparative-complacency}}

The model for bounded tolerance described above places a limit on
the increase of latency on a particular route. An alternative approach
would be to limit the increase in utility that alternative routes
gain over a particular route. In other words, the model developed
in Section \ref{sub:Modeling-limited-complacency} assumes that a
particular route flow would be complacent on its original route as
long as its own latency does not increase too much, \emph{while not
considering the possibility that the utilities of alternative routes
may have increased significantly}. To address this limitation, we
introduce a \emph{comparative tolerance} model and discuss the underlying
modeling assumptions.

We first assume for a given route $\route\in\routes$ with origin
$\origin_{\route}\in\origins$ and destination $\destination_{\route}\in\destinations$,
and assuming a tolerance scale factor $\alpha=0$ (no tolerance to
delays induced by cooperative flows), that the allowable difference
between the route's final latency and the final latency of all other
routes sharing the same origin and destination is the largest difference
in nominal latencies between itself and all other routes, or 0 if
the considered route has the smallest nominal latency. Then, if the
scale factor $\alpha$ is greater than 0, this allowable difference
is increased by $\alpha\origlatency_{\route}$. Mathematically, we
can express this condition as follows:

\begin{eqnarray*}
\latency_{\route}-\latency_{\bar{\route}}\le & \max\left(0,\max_{\tilde{\route}|\origin,\destination\in\tilde{\route},\tilde{\route}\neq\route}\left(\origlatency_{\route}-\origlatency_{\tilde{\route}}\right)\right)+\alpha\origlatency_{\route} & \forall\bar{\route}:\origin,\destination\in\bar{\route},\bar{\route}\neq\route\\
 &  & \forall\route:\origin,\destination\in\route,\forall\origin\in\origins,\forall\destination\in\destinations
\end{eqnarray*}


While this formulation more accurately captures the concept of traveler
behavior under improved comparative information about alternative
routes, it introduces many more constraints than the bounded tolerance
formulation in Section \ref{sub:Modeling-limited-complacency}. Additionally,
since the LHS of the constraint is a less-than inequality that contains
the subtraction of two functions of decision variables, common assumptions
on the latency functions will typically lead to a non-convex constraint.

One assumption that will guarantee convexity of the above constraints
is if all links have affine latency functions. It can be seen by considering
that the LHS is the summation of link latencies along a particular
route, and the RHS is a constant that can be computed a priori. In
Section \ref{sub:Linear-Latency} we will give a numerical example
of a simple network with linear latency functions, comparing the output
of our model assuming first simple bounded tolerance, and then considering
comparative tolerance.



\section{Formulating bounded tolerance as a convex optimization problem\label{sec:le:Convex-Optimization-Program}}

The preceding sections discussed a generic model for route-based flow
optimization on a flow network with mixed Lagrangian-Eulerian demands,
without identifying any specific flow model. In this section, we discuss
two types of flow models, horizontal queues and vertical queues. We
show for each case how the modeling assumptions can be made into convex
constraints, enabling one to solve the partial cooperation, bounded
tolerance model as a convex optimization problem. We begin our discussion
showing how vertical queues fit cleanly within our model (Section~\ref{sub:Vertical-Queueing}).
However, modeling horizontal queues (e.g. highway networks) requires
some additional theoretical setup. What has worked for modeling internet,
supply chains, etc. does not work for highway networks, as they are
nonlinear systems with non-convex constraints that depend on the density
of the links, rather than the flows. Its discussion constitutes the
bulk of the section (Section~\ref{sub:Horizontal-Queueing}).


\subsection{Vertical queues\label{sub:Vertical-Queueing}}

Several types of networks, such as communication networks or machine
queues (e.g. \cite{roughgarden2001stackelberg}), can model link
latencies as a function of the aggregated flow on the link. To contrast
with the model discussed in Section \ref{sub:Horizontal-Queueing},
we refer to such networks as \emph{vertical queues}. In this section,
we show an example of how the concepts of partial cooperation and
bounded tolerance can be modeled as a convex optimization program
for a specific class of vertical queues, and give a brief discussion
on how the results extend to a more general class of vertical queues.


\subsubsection{Example: M/M/1 queueing model\label{sub:M/M/1-Queueing-Model}}

A common way to model latencies for communication networks is the
M/M/1 queue~(e.g. \cite{aswani2011game}), which assumes Poisson
arrivals and exponential service times. On a link $\link\in\links$,
the average latency as a function of the rate of Poisson arrivals
(the flow) $\linkflow{}$ is given by the equation:

\begin{equation}
\latency_{\link}\left(\linkflow{}\right)=\frac{\occupationtime_{\link}}{\processtime_{\link}-\linkflow{}}\label{eq:vq}
\end{equation}


where $\occupationtime_{\link}$ is the occupation rate and $\processtime_{\link}$
is the processing rate. The flow on a link must be less than the processing
rate for the system to be stable. The function $\latency_{\link}\left(\cdot\right)$
is convex in $\left[0,\mu_{\link}\right)$. We can now substitute
Equation \eqref{eq:vq} into \eqref{eq:le} and \eqref{eq:limcomp-nonconvex}
to obtain the following program:

\begin{eqnarray}
\begin{array}{rcc}
\underset{\linkflow{\link}|\link\in\links,\routeflow{\route}|\route\in\routes}{\text{minimize}} &  & \sum\nolimits _{\link\in\links}\frac{\occupationtime_{\link}\linkflow{\link}}{\processtime_{\link}-\linkflow{\link}}\\
\text{subject to:}\\
 &  & \begin{aligned}\sum\nolimits _{\link\in\route}\frac{\occupationtime_{\link}}{\processtime_{\link}-\linkflow{\link}} & \le &  & \origlatency_{\route}\left(1+\complacencyscale\right) &  & \forall\route\in\routes\\
\linkflow{\link} & = &  & \sum\nolimits _{\routesthroughlink}\comprouteflow{\route}+\prevnclinkflow{\link} &  & \forall\link\in\links\\
\sum_{\routesthroughod}\comprouteflow{\route} & = &  & \odflow{\origin}{\destination} &  & \forall\origin\in\origins,\destination\in\destinations
\end{aligned}
\end{array}\label{eq:le-1}
\end{eqnarray}


where again, $\complacencyscale$ is given and corresponds to the
maximal threshold tolerable by users if Lagrangian (cooperative) demand
perturbs the nominal flow. This program is convex and can be solved
by standard convex solvers (with some algebraic manipulations for
disciplined convex programming solvers). Indeed, the objective is
the summation of convex functions, and the first constraint is a convex
inequality \emph{(less-than} inequality with a summation of convex
functions on the LHS).


\subsubsection{Class of convex vertical queues\label{sub:Class-of-Convex}}

This section shows that if all link latencies are convex, increasing
functions of flow~(e.g. following the modeling assumptions of \cite{roughgarden2001stackelberg}),
then the partial cooperation, bounded tolerance problem is guaranteed
to be convex.

From the discussion in Section \ref{sub:M/M/1-Queueing-Model}, we
can generalize the class of latency functions for vertical queues,
which lead to a convex formulation. In Equation \eqref{eq:le}, only
the objective contains latency terms, and in Equation \eqref{eq:limcomp-nonconvex},
the LHS of the inequality contains latency terms. Therefore, we need
to verify the convexity of the objective and the bounded tolerance
constraints.

A well known-result of convex analysis is that the summation of convex
functions preserves convexity. Therefore, the convexity of the bounded
tolerance constraint is guaranteed if the latency function on every
link is convex. Additionally, for a link $\link\in\links$, the total
link latency, $\linkflow{\link}\latency_{\link}\left(\linkflow{\link}\right)$,
is convex from the assumptions on $\latency_{\link}$, and therefore
the sum of all total link latencies (which is the objective) is guaranteed
to be convex.


\subsection{Horizontal queues\label{sub:Horizontal-Queueing}}

A standard assumption in transportation networks is that latencies
are not determined uniquely by the flow on the link, but rather how
densely populated the link is. Such latency models are referred to
as \emph{horizontal queue}s, as the congestion on a link occupies
physical space which may propagate in the horizontal direction. In
this section, we show how the partial cooperation and bounded tolerance
models can be extended to horizontal queues. First we develop the
relationship between link flow and link density, the resulting latency
model, and how a convex optimization problem can be formulated for
networks with horizontal queues.


\subsubsection{Link model\label{sub:Link-model}}


\paragraph{Constraining flow by link densities}

For each horizontal queuing link $\link\in\links$, in addition to
having a link flow $\linkflow{\link}$, a horizontal queue link also
has a density of vehicles $\density_{\link}$, expressing the number
of vehicles occupying a link divided by the length $\length_{\link}$.
Relating the density of a link to its flow, each link also has a trapezoidal
fundamental diagram specified by three parameters: free-flow velocity
$\ffspeed_{\link}$, congestion velocity $\congspeed_{\link}$, and
max flow $\flowmax{\link}$.
From these parameters, one can compute the critical density $\densitycrit_{\link}$
and jam density $\densitymax_{\link}$. Given that we are assuming
the network is in equilibrium, then outflow must equal inflow for
each link~(see \cite{gomes2008behavior} for a detailed analysis
of horizontal queue equilibria). Therefore, only need to consider
the single variable $\linkflow{\link}$ when analyzing flow on a link,
rather than considering both the inflow and outflow of a link. We
express the $\density_{\link}$ (as traditionally assumed by the LWR
equation~\cite{daganzo1994cell,daganzo1995cell,lighthill1955kinematic,richards1956shock}),
we have two coupled variables $\linkflow{\link}$ and $\density_{\link}$,
with the following constraints:

\begin{eqnarray}
\linkflow{\link} & \le & \ffspeed_{\link}\density_{\link}\label{eq:constraint-demand}\\
\linkflow{\link} & \le\congspeed_{\link} & \left(\densitymax_{\link}-\density_{\link}\right)\label{eq:constraint-supply}\\
0\le\linkflow{\link} & \le & \flowmax{\link}\label{eq:constraint-maxflow}
\end{eqnarray}


where \eqref{eq:constraint-demand} restricts the outflow of link
$\link$, \eqref{eq:constraint-supply} restricts the inflow,
and \eqref{eq:constraint-maxflow} is a physical capacity of the
link. These constraints are a relaxation of the fundamental diagram,
initially introduced by \cite{gomes2006optimal}.


\paragraph{Latencies}

For a link $\link\in\links$, the latency $\ell_{\link}$ is obtained
by multiplying the length and velocity of the link, where the velocity
of the link is a function of both flow and density\@. With a notational
change (the latency now depending on two variables), the latency function
is given by:

\[
\ell_{\link}\left(\flow,\density\right)=\frac{\length_{\link}\density}{\flow}
\]


and the total latency $\linkflow{\link}\latency_{\link}\left(\linkflow{\link},\density_{\link}\right)$
on a link is given simply by the number of vehicles on a link, or
$\length_{\link}\density$. Note that a nominal link latency must
be determined from both a nominal flow and nominal density, requiring
more information than the point queue model, which only needs nominal
link flows.


\subsubsection{Relaxation of Junction Model\label{sub:Relaxation-of-Junction}}

In order to guarantee uniqueness of solutions of junction problems
in LWR networks, it is common to assume that the sum of flows across
junctions is maximal, while respecting the prescribed turning ratios.
\cite{daganzo1995cell} describes a junction model for 2-to-1 merges
and 1-to-2 diverges tailored to the CTM model, while \cite{garavello2006traffic}
describe a more general junction model allowing $n$-to-$m$ merges
for the continuous LWR network model, which includes the Daganzo model
as a special case with triangular fundamental diagrams and limited
merge/diverge types. We refer to the flow-maximizing junction models
as the \emph{unrelaxed }junction model, and the flow-density relationship
in Section~\ref{sub:Link-model} as the \emph{relaxed} junction model
as it does not include a flow maximization condition.

One technical reason why the relaxed model is used is that a flow-maximization
condition would lead to a non-convex problem formulation. Another
argument that can be made is that for certain junction types, some
\emph{split-ratio vector} or \emph{priority vector }(see \cite{Coclite2002})
may exist that would lead to the flow solution given by the optimization
problem. Therefore, since this problem has no fixed split-ratios,
it can be considered a free variable and the optimization problem
has discovered one of the many possible solutions to some junction.
This argument has limits, as there is no such free parameter for 1-to-1
junction types, for instance.

There have been methods proposed for dealing with the implicit ``car
holding'' issue introduced from the relaxation, such as adding penalty
terms in the objective~(see \cite{ziliaskopoulos2000linear}), but
we do not consider these in our analysis.


\subsubsection{Optimization program\label{sub:Optimization-program}}

For the horizontal queues network, we can now express the total latency
minimization problem expressed in Section \eqref{sub:Modeling-limited-complacency}:

\begin{align}
\begin{array}{rcc}
\underset{\linkflow{\link},\density_{\link}|\link\in\links,\routeflow{\route}|\route\in\routes}{\text{minimize}} &  & \sum\nolimits _{\link\in\links}\length_{\link}\density_{\link}\\
\text{subject to:}\\
 &  & \begin{aligned}\sum\nolimits _{\link\in\route}\frac{\length_{\link}\density_{\link}}{\linkflow{\link}} & \le &  & \left(1+\complacencyscale\right)\origlatency_{\route} &  & \forall\route\in\routes\\
\linkflow{\link} & \le &  & \ffspeed_{\link}\density_{\link} &  & \forall\link\in\links\\
\linkflow{\link} & \le &  & \congspeed_{\link}\left(\densitymax_{\link}-\density_{\link}\right) &  & \forall\link\in\links\\
0 & \le &  & \linkflow{\link}\le\flowmax{\link} &  & \forall\link\in\links\\
\linkflow{\link} & = &  & \sum\nolimits _{\routesthroughlink}\comprouteflow{\route}+\prevnclinkflow{\link} &  & \forall\link\in\links\\
\sum_{\routesthroughod}\comprouteflow{\route} & = &  & \odflow{\origin}{\destination} &  & \forall\origin\in\origins,\destination\in\destinations
\end{aligned}
\end{array}\label{eq:hqproblem}
\end{align}


This formulation is not convex, specifically the bounded tolerance
constraint is not convex. There is a superseding problem with the
formulation, that the bounded tolerance constraints and outflow constraints
($\linkflow{\link}\le\congspeed_{\link}\left(\densitymax_{\link}-\density_{\link}\right)$)
are guaranteed to be non-binding. Several of the constraints can be
shown to be non-binding by observing that a solution must satisfy
$\density_{\link}=\frac{\linkflow{\link}}{\ffspeed_{\link}}$. We
prove that next.
\begin{lem}
\label{lem:rho}If the solution $\veccomprouteflow^{*},\vecdensity^{*}$
is optimal for Problem~\ref{eq:hqproblem}, then $\forall\link\in\links$:
\textup{$\density_{\link}^{*}=\frac{\linkflow{\link}^{*}}{\ffspeed_{\link}}$}\end{lem}
\begin{proof}
Assume $\exists\density_{\link}>\frac{\linkflow{\link}^{*}}{\ffspeed_{\link}}$.
Reducing $\density_{\link}$ to $\frac{\linkflow{\link}^{*}}{\ffspeed_{\link}}$
only decreases the LHS of the first constraint in Problem~\eqref{eq:hqproblem}.
The second constraint becomes an equality by construction. The RHS
of the third constraint increases. Since the flow terms are not changed,
we see that the feasibility of the problem is maintained. Additionally,
the objective strictly decreases, thus proving that a solution with
such a $\density_{\link}$ is sub-optimal.
\end{proof}
We can now simplify Problem~\ref{eq:hqproblem} by substituting in
the value of $\vecdensity$ from Lemma~\ref{lem:rho}, and using
the following notational change for the parameters $\length_{\link}$
and  $a_{\link}=\frac{L_{\link}}{v_{\link}}$:

\begin{eqnarray}
\min_{\linkflow{\link}|\link\in\links,\comprouteflow{\route}|\route\in\routes} & \sum_{\link\in\mathcal{L}}a_{\link}\linkflow{\link}\label{eq:rhosimp}\\
\text{subject to:}\nonumber \\
 & \sum_{\link\in\route}\length_{\link}\ffspeed_{\link}\le\left(1+\complacencyscale\right)\origlatency_{\route} & \forall\route\in\routes\nonumber \\
 & \linkflow{\link}\le\ffspeed_{\link}\left(\frac{\linkflow{\link}}{\ffspeed_{\link}}\right) & \forall\link\in\links\nonumber \\
 & \linkflow{\link}\le\congspeed_{\link}\left(\densitymax_{\link}-\frac{\linkflow{\link}}{\ffspeed_{\link}}\right) & \forall\link\in\links\nonumber \\
 & 0\le\linkflow{\link}\le\flowmax{\link} & \forall\link\in\links\nonumber \\
 & \linkflow{\link}=\sum\nolimits _{\routesthroughlink}\comprouteflow{\route}+\prevnclinkflow{\link} & \forall\link\in\links\nonumber \\
 & \sum_{\routesthroughod}\comprouteflow{\route}=\odflow{\origin}{\destination} & \forall\origin\in\origins,\destination\in\destinations\nonumber \\
\nonumber 
\end{eqnarray}


We can now detect non-binding constraints easily. The first constraint
in Problem~\eqref{eq:rhosimp} must be satisfied because the
LHS is the free-flow travel time of the route and is minimal, while
the RHS must be greater or equal to free-flow (keeping in mind $\alpha\ge0$).
The second constraint is always an equality, by Lemma~\ref{lem:rho}.
The third constraint is guaranteed from the assumption of a trapezoidal
fundamental diagram. The simplified problem is now:

\begin{eqnarray}
\min_{\linkflow{\link}|\link\in\links,\comprouteflow{\route}|\route\in\routes} & \sum_{\link\in\mathcal{L}}a_{\link}\linkflow{\link}\label{eq:rhosimp-1}\\
\text{subject to:}\nonumber \\
 & 0\le\linkflow{\link}\le\flowmax{\link} & \forall\link\in\links\nonumber \\
 & \linkflow{\link}=\sum\nolimits _{\routesthroughlink}\comprouteflow{\route}+\prevnclinkflow{\link} & \forall\link\in\links\nonumber \\
 & \sum_{\routesthroughod}\comprouteflow{\route}=\odflow{\origin}{\destination} & \forall\origin\in\origins,\destination\in\destinations\nonumber 
\end{eqnarray}


The above problem is now in a linear program formulation. We now show
that the concept of noncooperative flow can be replaced by a capacity
reduction on all the links. Let us rework some of the expressions
in terms of the cooperative and noncooperative vehicles:

\begin{eqnarray}
\min_{\comprouteflow{\route}|\route\in\routes} & \sum_{\link\in\links}a_{\link}\prevnclinkflow{\link}+\sum_{\link\in\links}a_{\link}\sum_{\routesthroughlink}\comprouteflow{\route}\label{eq:subbed}\\
\text{subject to:}\nonumber \\
 & -\prevnclinkflow{\link}\le0\le\sum_{\routesthroughlink}\comprouteflow{\route}\le\flowmax{\link}-\prevnclinkflow{\link} & \forall\link\in\links\nonumber \\
 & \sum_{\routesthroughod}\comprouteflow{\route}=\odflow{\origin}{\destination} & \forall\origin\in\origins,\destination\in\destinations\nonumber 
\end{eqnarray}


We can now simplify further. The first term in the objective is constant,
since $\noncomprouteflow{\route}$ is not a decision variable. Then,
the second constraint can be simplified by introducing a reduced capacity
constant, $\bar{\flow}_{\link}^{\max}=\flowmax{\link}-\prevnclinkflow{\link}$.
If we drop the \emph{cooperative} pretense from the decision variable,
then we have reduced the problem to a modified capacity, constant
latency, Lagrangian system optimal problem, which is simplified and
linear:

\begin{eqnarray}
\min_{\routeflow{\route}|\route\in\routes} & \sum_{\link\in\links}a_{\link}\sum_{\routesthroughlink}\routeflow{\route}\label{eq:subbed-1}\\
\text{subject to:}\nonumber \\
 & 0\le\sum_{\routesthroughlink}\routeflow{\route}\le\bar{\flow}_{\link}^{\max} & \forall\route\in\routes\nonumber \\
 & \sum_{\routesthroughod}\routeflow{\route}=\odflow{\origin}{\destination} & \forall\origin\in\origins,\destination\in\destinations\nonumber 
\end{eqnarray}

\begin{lem}
\label{lem:main}Let $\vecrouteflow^{*}=\left\{ \routeflow{\route}^{*}:\route\in\routes\right\} $
be a solution to Problem~\eqref{eq:subbed-1}. Then

\begin{eqnarray*}
\veccomprouteflow' & = & \vecrouteflow^{*}\\
\density'_{\link} & = & \frac{\prevnclinkflow{\link}+\sum_{\routesthroughlink}\routeflow{\route}^{*}}{\ffspeed_{\link}}
\end{eqnarray*}


is a solution to Problem \eqref{eq:hqproblem}.\end{lem}
\begin{proof}
Using Lemma \ref{lem:rho}, the equality $\linkflow{\link}=\prevnclinkflow{\link}+\sum_{\routesthroughlink}\comprouteflow{\route}$,
and the variable name substitution made in Problem \ref{eq:subbed-1},
the result follows immediately.\end{proof}
\begin{cor}
\label{cor:An-optimal-solution}An optimal solution to Problem~\eqref{eq:hqproblem}
is a feasible solution of the unrelaxed junction model in Section~\ref{sub:Relaxation-of-Junction}.\end{cor}
\begin{proof}
Since the flow on every link is in free flow ($\linkflow{\link}=\ffspeed_{\link}\density_{\link},\forall\link\in\links$),
the supply $\sum_{\link\in\inlinks{\junction}}\ffspeed_{\link}\density_{\link}$
at every junction $\junction\in\junctions$ is equal to the flow across
the junction $\sum_{\link\in\inlinks{\junction}}\linkflow{\link}$,
and is therefore maximal.
\end{proof}
This corollary shows that solving for the static social optimum on
networks with horizontal queues does not encounter the non-convexity
issues typically associated with the CTM constraints in dynamic traffic
problems. For instance, \cite{ziliaskopoulos2000linear} uses the
relaxed junction model (which allows ``car-holding'') that we present
in Section~\ref{sub:Link-model} to solve the single destination
social optimum problem as a linear program, and \cite{gomes2006optimal}
use a relaxed junction model to solve an optimal ramp metering problem
as a linear program (with a zero-relaxation gap under certain conditions).

Commonly considered problems in traffic assignment such as social
optimum for purely Lagrangian flow ($\prevnclinkflow{\link}=0,\forall\link\in\links$)
and purely Eulerian flow ($\comprouteflow{\route}=0,\forall\route\in\routes$)
serve as special cases of Corollary~\ref{cor:An-optimal-solution}
and therefore an optimal solution can be found for both problems for
the unrelaxed junction model by solving the linear program in Problem~\eqref{eq:subbed-1}. 


\subsubsection{Limiting deviations in density\label{sub:Limiting-Deviations-in}}

There are limitations in the expressiveness of the current horizontal
queues model under total latency minimization. To circumvent these
issues, this section proposes the addition of constraints that restrict
the allowable densities to be within the locality of the nominal densities
that are used to compute nominal latencies.

The purpose of these constraints is to prevent the optimization program
from setting all links to be in the free-flow state, which has the
negative effect of over-simplifying the model developed here-within
(Section \eqref{sub:Optimization-program}). Instead, total latencies
across the network can be minimized \emph{while considering likely
congestion patterns}. To motivate the usefulness of such a model,
one can make a physical argument that rerouting may only cause bounded
deviations in the density, and that congestion may not be cleared
due to rerouting because of additional issues such as weaving or the
physical road conditions.

To restrict the densities to only take certain values, we require
that each link $\link\in\links$, includes an upper and lower density
bound, $\densityupper_{\link}$ and $\densitylower_{\link}$ respectively.
We append to the program in Equation \eqref{eq:hqproblem}, the set
of constraints bounding the allowable densities:

\begin{eqnarray*}
\densitylower_{\link} & \le\density_{\link}\le\densityupper_{\link} & \forall\link\in\links
\end{eqnarray*}


In Section~\ref{sec:le:Numerical-Results}, we show an example of a
network with horizontal queues with density bounds that has the bounded
tolerance constraint as a tight constraint. This demonstrates that
bounding the allowable densities can capture the characteristics of
bounded tolerance for networks with horizontal queues.


\subsection{Algorithm for data preconditioning\label{sub:Preparing-Input-Data}}

If input data into our problem is taken from a physical network with
inherent sources of noise, it is likely that there will be a number
of conditions that will cause the raw data to be incompatible with
the problem constraints, thus making the problem infeasible. For instance,
a link's estimated density may not lie within the fundamental diagram
constraints in Section \ref{sub:Link-model}, or there may not be
exact mass balance across junctions. If the estimates from the stationary
sensors are reasonable, then these constraint violations will not
be severe, but even small deviations will render the optimization
program infeasible. Therefore the input data must be filtered to be
preconditioned to meet the requirements. We propose an optimization
program formulation.

The constraints that concern noncooperative flow are the following:

\begin{eqnarray}
 & \linkflow{\link}\le\ffspeed_{\link}\density_{\link} & \forall\link\in\links\label{eq:prepconstraintsbegin}\\
 & \linkflow{\link}\le\congspeed_{\link}\left(\densitymax_{\link}-\density_{\link}\right) & \forall\link\in\links\\
 & 0\le\linkflow{\link}\le\flowmax{\link} & \forall\link\in\links\\
 & \linkflow{\link}=\sum_{\routesthroughlink}\routeflow{\route} & \forall\link\in\links\label{eq:prepconstraints}
\end{eqnarray}


These constraints are all convex (indeed, linear) in the decision
variables $\linkflow{\link},\routeflow{\route}$. Then, let $\hat{\flow}_{\link},\hat{\density}_{\link}$
be the input flow and density respectively on link $\link\in\hat{\links}$,
where $\hat{\links}\subseteq\links$ are the links with input data
available. Our objective will be to minimize some definition of distance
from the input data to the selected data that violates none of the
above constraints. If we select as the distance measurement the \emph{n-norm},
$n\ge1$, then we have the following convex optimization program for
obtaining amenable input data:

\begin{eqnarray*}
\underset{\linkflow{\link},\density_{\link}:\link\in\links}{\mytext{minimize}} & \sum_{\link\in\inputlinks}\left\Vert \hat{\flow}_{\link}-\linkflow{\link}\right\Vert _{n}+\left\Vert \hat{\density}_{\link}-\density_{\link}\right\Vert _{n}\\
\text{subject to:} & \text{Constraints }\eqref{eq:prepconstraintsbegin}-\eqref{eq:prepconstraints}
\end{eqnarray*}


The result of the optimization problem is a set of route-based flows
$\left\{ \routeflow{\route}:\route\in\routes\right\} $. Finally,
the route flows would then be partitioned into both cooperative and
noncooperative flows, which then gets the data in a suitable format.

While the formulation presented specifically discusses horizontal
queue constraints, the same methods can be extended to other problems
with convex constraints, such as M/M/1 queues presented in Section~\ref{sub:M/M/1-Queueing-Model}.



\section{Numerical results}
\label{sec:le:Numerical-Results}

We demonstrate the highly practical nature of our work by applying
the model to two different problems. We focus on a multi-destination
network of horizontal queues. This problem demonstrates the generality
of our method to the multi-commodity case and the ability to solve
real-world transportation planning problems on a regional level. First,
we demonstrate the simplicity of the model on a small network of vertical
queues, to which we apply both models of tolerance and compare the
benefits gained by re-routing.


\subsection{Linear Latency\label{sub:Linear-Latency}}

\begin{wrapfigure}{o}{0.25\columnwidth}%
\begin{centering}
\includegraphics[width=0.25\columnwidth]{previous-articles/le/le-figures/simple-network}
\par\end{centering}

\protect\caption[Network diagram for 2-link communication network routing example.]{Network diagram\label{fig:Simple-network-diagram}}
\end{wrapfigure}%
Figure~\ref{fig:Simple-network-diagram} depicts an illustrative,
two parallel routes network. Flow enters at the source and exits at
the sink and can travel along either the ``left'' route or the ``right''
route. Each link $\link\in\links$ has a linear latency function $\latency_{\link}\left(\linkflow{}\right)=a_{\link}\linkflow{}+b_{\link}$.
The link properties given in Figure~\ref{tab:Simple-network-parameters}
show that the left link has a lower \emph{zero-flow} latency than
the right link, but has a higher marginal cost per unit flow. As the
left route becomes more congested, its latency will eventually increase
until the right route has equal utility. From a user equilibrium viewpoint,
as the network is loaded with additional flow, the latencies across
the two routes will remain the same. But from a social optimum viewpoint,
additional flow will always be routed to the right route since it
will always have a lower marginal cost than the left route.
\begin{table}
\centering
\subfloat[\label{tab:Simple-network-parameters}]{
\begin{tabular}{|c|c|c|c|}
\hline 
Name & $a$ $\left(\frac{s^{2}}{\text{units}}\right)$ & $b$ $\left(s\right)$ & $\flowmax{}$ $\left(\frac{\text{units}}{s}\right)$\tabularnewline
\hline 
\hline 
source & 1 & 0 & 1\tabularnewline
\hline 
sink & 1 & 0 & 1\tabularnewline
\hline 
left & 1 & 0 & 1\tabularnewline
\hline 
right & 0.5 & 0.5 & 1\tabularnewline
\hline 
\end{tabular}
}\hfill%
\subfloat[\label{tab:Demands}]{
\begin{tabular}{|c|c|c|}
\hline 
Type & Description & $\flow$ ($\frac{\text{cars}}{\text{s}}$)\tabularnewline
\hline 
\hline 
O-D (Lagrangian) & source-sink & 0.8\tabularnewline
\hline 
Link (Eulerian) & source & 0.2\tabularnewline
\hline 
Link (Eulerian) & sink & 0.1\tabularnewline
\hline 
Link (Eulerian) & left & 0.1\tabularnewline
\hline 
Link (Eulerian) & right & 0.2\tabularnewline
\hline 
\end{tabular}
}
\caption{Summary of illustrative network properties. \textbf{\ref{tab:Simple-network-parameters}}: Link-level input parameters. \textbf{\ref{tab:Demands}:} Network-level input demands}
\end{table}


As described in Figure~\ref{tab:Demands}, we assume the network
is loaded with both cooperative and noncooperative flow. There is
a total of 0.2 units-per-second of noncooperative flow, with 0.1 units-per-second
of flow on both the left and right link. In addition, there is 0.8
units-per-second of flow on the network, which we assume is initially
distributed amongst the left and right routes in a manner that achieves
user equilibrium.

We now show how our route optimization framework can be applied to
this network to optimally route the cooperative flow. Results are
given over a range of bounded rationality scale factor to show the
sensitivity of our results to the scale factor. Since the latency
functions are linear, both the standard bounded rationality model
and comparative bounded rationality model are solved using well-developed
and highly efficient convex optimization tools~(see \cite{cvxpy}).
In addition to comparing the two models against each other, we compare
them both against the Stackelberg game solution of the routing problem.
The Stackelberg game solution gives a minimum social cost with the
assumption that the noncooperative flow will be routed in a user equilibrium
manner. Stackelberg analysis is only possible in the case when origin-destination
demands can be uniquely determined for all users of the network, which
holds for our simple network. This does not hold in general, and this
case will studied subsequently (Section~\ref{sub:Horizontal-Queueing-Network:}).
\begin{figure}
\centering%
\subfloat{%
\includegraphics[scale=0.7]{previous-articles/le/le-figures/complacency_models_simple_network}%
\label{fig:Route-latencies}
}\hfill%
\subfloat{%
\includegraphics[scale=0.7]{previous-articles/le/le-figures/complacency_models_simple_network_total_latency}
\label{fig:Total-latencies}
}%
\caption[Comparison of simple bounded tolerance and comparative tolerance.]{Comparison of simple bounded tolerance
and comparative tolerance. \textbf{\ref{fig:Route-latencies}:} Route
latencies. Comparative tolerance allows smaller deviations in route
latencies than bounded tolerance. \textbf{\ref{fig:Total-latencies}:}
Total latencies. The total latencies decrease more slowly with the
comparative tolerance model versus the bounded tolerance model. The
total route flows approach the Stackelberg equilibrium as the tolerance
scale factor goes to infinite.}%
\label{fig:Comparison-of-simple}
\end{figure}


Figure~\ref{fig:Comparison-of-simple} summarizes the numerical results
on the simple network. The route latencies as a function of the bounded
rationality scale factor are shown in Figure~\ref{fig:Route-latencies},
while total latencies are shown in Figure~\ref{fig:Total-latencies}.
As expected, as the bounded rationality scale factor increases, so
do the benefits of re-routing. Additionally, the comparative bounded
rationality model improves at a slower rate than the standard bounded
rationality model. This is due to the fact that the comparative model
permits the right route to be ``aware'' of the latency improvements
on the left route, while the standard model only limits deviations
in route latencies in comparison to a route's individual \emph{nominal}
latency and ignores the improvements on the left route.

The results tell us that as the scale factor increases, the model
converges to the Stackelberg solution. It may appear counter-intuitive
that the model with inherently no tolerance factor could perform better
than the tolerance models. The explanation is that the tolerance models
are overly-conservative due to the assumption that no noncooperative
flow changes routes, and the routing strategy will not drastically
improve one route over another route. On the other hand, the Stackelberg
solution shifts all noncooperative flow to the left route and the
cooperative flow accommodates this shift in a socially optimal manner.
Since all noncooperative flow is on a single route and improves upon
its nominal latency, discrepancies in route latencies are no longer
a behavioral issue, allowing the Stackelberg to be as liberal as necessary
with latency increases on the right route.


\subsection{Horizontal queueing network\label{sub:Horizontal-Queueing-Network:}}

\begin{figure}[h]
\centering
\subfloat[Multiple-destination network with horizontal queues. There are many overlapping routes between \emph{Source} and \emph{Sink A}, while \emph{Sink B} and \emph{Sink C} are origin-destinations which have demands on the same network as \emph{Sink A} demands.]{
\includegraphics[width=0.4\columnwidth]{previous-articles/le/le-figures/HighwayNetworkDos}%
\label{fig:Multiple-destination-network-wit}
}\hfill%
\subfloat[Total latency on network of horizontal
queues as a function of tolerance scale.]{
\includegraphics[width=0.5\columnwidth]{previous-articles/le/le-figures/hqsimresults}%
\label{fig:Total-latency-on}
}%
\end{figure}

As discussed in Section~\ref{sub:Horizontal-Queueing}, given the
nonlinear dynamics of horizontal queueing cells, modeling horizontal
queues is in general a more difficult process than vertical queues.
It is also important to consider a more general network than the compact
one in Section~\ref{sub:Linear-Latency}, one with multiple destinations,
and therefore multiple Lagrangian demand types. In this section, we
model a mid-sized multi-destination network of horizontal queues within
the partial cooperation, bounded tolerance framework. We follow with
numerical results on how the routing strategies change with respect
to the parameters of the tolerance model.


\subsubsection{Network properties and demands\label{sub:Network-properties-and}}

Figure~\ref{fig:Multiple-destination-network-wit} shows a topological
description of the network. Since only \emph{Sink A }can be reached
through multiple routes, the algorithm only decides how to partition
the demand across the routes originating from \emph{Source} and leading
to \emph{Sink A}. The algorithm will take as input some network and
link level properties, as recorded in Table~\ref{tab:Multi-destination-network-proper}.
The nominal state given in Table~\ref{tab:prophq} shows that links
3, 8 and 9 were heavily congested, while the other links were close
to free flow. Furthermore, as discussed in Section~\ref{sub:Limiting-Deviations-in},
the densities must have more constraints than just the fundamental
diagram constraints (Equations~\eqref{eq:constraint-demand}-\eqref{eq:constraint-maxflow}).
Table~\ref{tab:prophq} tells us that this problem assumes that links
do not shift from their nominal state (links with a nominal free flow/congestion
state must maintain this state).
\begin{table}
\subfloat[\label{tab:prophq}]{%
\begin{tabular}{|c|c|c|c|c|c|c|c|c|c|c|c|c|c|c|c|c|}
\cline{1-1} \cline{3-17} 
{\scriptsize{}Name} &  & {\scriptsize{}0} & {\scriptsize{}1} & {\scriptsize{}2} & {\scriptsize{}3} & {\scriptsize{}4} & {\scriptsize{}5} & {\scriptsize{}6} & {\scriptsize{}7} & {\scriptsize{}8} & {\scriptsize{}9} & {\scriptsize{}10} & {\scriptsize{}11} & {\scriptsize{}12} & {\scriptsize{}13} & {\scriptsize{}14}\tabularnewline
\cline{1-1} \cline{3-17} 
{\scriptsize{}Length ($m$)} &  & {\scriptsize{}0.5} & {\scriptsize{}0.1} & {\scriptsize{}1.0} & {\scriptsize{}1.5} & {\scriptsize{}0.3} & {\scriptsize{}0.1} & {\scriptsize{}0.6} & {\scriptsize{}0.4} & {\scriptsize{}1.0} & {\scriptsize{}1.5} & {\scriptsize{}1.0} & {\scriptsize{}0.7} & {\scriptsize{}0.8} & {\scriptsize{}1.0} & {\scriptsize{}0.1}\tabularnewline
\cline{1-1} \cline{3-17} 
{\scriptsize{}Nom. flow ($\frac{\text{units}}{s}$)} &  & {\scriptsize{}0.15} & {\scriptsize{}0.05} & {\scriptsize{}0.03} & {\scriptsize{}0.02} & {\scriptsize{}0.05} & {\scriptsize{}0.05} & {\scriptsize{}0} & {\scriptsize{}0.02} & {\scriptsize{}0.01} & {\scriptsize{}0.02} & {\scriptsize{}0} & {\scriptsize{}0.02} & {\scriptsize{}0.01} & {\scriptsize{}0.05} & {\scriptsize{}0.05}\tabularnewline
\cline{1-1} \cline{3-17} 
{\scriptsize{}Nom. density ($\frac{\text{units}}{m}$)} &  & {\scriptsize{}0.15} & {\scriptsize{}0.05} & {\scriptsize{}0.03} & {\scriptsize{}4.99} & {\scriptsize{}0.05} & {\scriptsize{}0.05} & {\scriptsize{}0} & {\scriptsize{}0.02} & {\scriptsize{}4.99} & {\scriptsize{}4.99} & {\scriptsize{}0} & {\scriptsize{}0.02} & {\scriptsize{}0.01} & {\scriptsize{}0.05} & {\scriptsize{}0.05}\tabularnewline
\cline{1-1} \cline{3-17} 
{\scriptsize{}State constraint} &  & {\scriptsize{}FF} & {\scriptsize{}FF} & {\scriptsize{}FF} & {\scriptsize{}Cong} & {\scriptsize{}FF} & {\scriptsize{}FF} & {\scriptsize{}FF} & {\scriptsize{}FF} & {\scriptsize{}Cong} & {\scriptsize{}Cong} & {\scriptsize{}FF} & {\scriptsize{}FF} & {\scriptsize{}FF} & {\scriptsize{}FF} & {\scriptsize{}FF}\tabularnewline
\cline{1-1} \cline{3-17} 
\end{tabular}

}

\subfloat[\label{tab:demandhq}]{%
\begin{tabular}{|c|c|c|c|c|}
\cline{1-1} \cline{3-5} 
Sink Dest. &  & A & B & C\tabularnewline
\cline{1-1} \cline{3-5} 
Flow ($\frac{\text{users}}{s}$) &  & 0.5 & 0.5 & 0.5\tabularnewline
\cline{1-1} \cline{3-5} 
\end{tabular}

}

\protect\caption{\label{tab:Multi-destination-network-proper}Multi-destination network
properties. \textbf{\ref{tab:prophq}:} Link properties, including
nominal state. \textbf{\ref{tab:demandhq}: }demand input into network}
\end{table}
 Table~\ref{tab:demandhq} tells us that there is 0.5 $\frac{\text{users}}{sec}$
demand between all origin-destination pairs on the network. For simplicity,
we assume that all demand is cooperative as well to focus analysis.


\subsubsection{Numerical results\label{sub:Numerical-results}}

The results of our numerical calculations are summarized in Figure~\ref{fig:Total-latency-on}.
As supported by the results for vertical queues in Section~\ref{sub:Linear-Latency},
the relief of network congestion is greater the more tolerance is
assumed in the users. Additionally, it is noted that the network does
not immediately push into free flow (social optimum), but rather decongests
links to an amount dependent on the tolerance scale factor. This is
a desirable behavior of the model, as it is not reasonable to assume
that congestion can be completely avoided just through re-routing
schemes. Lastly, we see the intuitive result that the bounded tolerance
model will converge to the more familiar social optimum as the scale
factor increases.



\section{Summary of Results}
\label{sec:le:Conclusion}

We have presented a framework rerouting flow in an socially optimal
way with mixed Lagrangian-Eulerian information. The cooperative flow
has known nominal routes, while the noncooperative flow has known
flow counts across links. In order to anticipate network conditions
for all users after re-routing has been applied, the model combines
the two types of information in a complementary way; by only allowing
the cooperative flow to change routes, we have removed the necessity
of having origin-destination demand information for all users on the
network. Furthermore, by looking at the static flow problem, we can
study practical networks with multiple origins and multiple destinations,
where dynamic multi-commodity problems often suffer from intractability
issues.

The framework also addresses the behavioral nature of the noncooperative
users, which we call \emph{bounded tolerance }and \emph{comparative
tolerance}, by only allowing perturbations of the nominal state of
the network that boundedly impact the noncooperative flow in a negative
manner. The tolerance model comes about as a response to the lack
of origin-destination information that does not permit the game-theoretic
Stackelberg game analysis, but does allow us to require only Eulerian
information across the majority of the network. We show that the comparative
tolerance model will in general limit network latency improvements
more so than the bounded tolerance model, but since the comparative
tolerance model allows individual routes to compare latencies with
other routes, it is arguably a more accurate model of noncooperative
flow behavior.

By taking a convex optimization-based approach, the framework is shown
to efficiently solve many classes of network flow problems. The horizontal
queue, highway network problem can be modeled as a convex optimization
program, which permits one to study highway networks of practical
size. The multi-destination network of horizontal queues gives an
example of how the framework can be applied to multi-commodity type
networks such as highways with multiple onramps and off-ramps. A live
data feed of Lagrangian GPS sensors and Eulerian loop detectors, in
conjunction with the data pre-conditioning algorithm, would enable
the framework to run in an ``online'' sense, and provide automatic,
daily routing advice for a traffic management agency during rush hour
periods. We conclude that the partial cooperation, bounded tolerance
model can allow a traffic management operator to make beneficial re-routing
decisions with much less origin-destination demand input required. 
