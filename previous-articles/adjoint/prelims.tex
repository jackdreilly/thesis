We consider the non-linear conservation equation of the form:
\begin{equation}
\partial_t\rho\left(t,x\right)+\partial_x f\left(\rho\left(t,x\right)\right)=0 \quad (t,x) \in \R^+ \times \R\label{eq:cons}
\end{equation}
where $\rho=\rho(t,x) \in \; \R^+$ is the scalar conserved quantity and $f:\R^+\rightarrow \R^+$ is a Lipschitz continuous flux function~\cite{Bressan2000Hyperbolic}. Throughout the article we suppose that $f$ is a concave function. \\
The Cauchy problem to solve is then 
\begin{equation}
	\label{eq:CP}
		\left\{
		\begin{array}{ll}
		\partial_t \rho+ \partial_x f(\rho) =0, & (t,x)\in\R^+\times \R,\\
		\rho(0,x)=\initstate(x), & x \in \R\\
		\end{array}
		\right.
\end{equation}
where $\initstate(x)$ is the initial condition.
It can be shown that there exists a unique weak entropy solution for the Cauchy problem\eqref{eq:CP},  as described in Definition \ref{def:weak-sol}. 
\begin{defn}\label{def:weak-sol}
A function $\rho \in \; \mathcal{C}^0(\R^+; \mathbf{L}^1_{loc}\cap \mathbf{BV})$ is an admissible solution to \eqref{eq:CP} if $\rho$ satisfies the Kru\v{z}hkov entropy condition \cite{Kruzhkov} on $(\R^+\times\R)$, i.e.,for every $k\in \R$ and for all $\varphi \in \mathcal{C}^1_c(\R^2;\R^+),$
\begin{eqnarray}
\label{eq:kruzhkov}
	&\int_{\R^+}\int_{\R}(\modulo{\rho -k}  \partial_t \varphi + \sgn{(\rho-k }) (f(\rho)-f(k))\partial_x\varphi)dxdt& \nonumber\\
	 &+\int_{\R}\modulo{\initstate-k}\varphi(0,x)dx\geq 0.&
\end{eqnarray} 
\end{defn}
For further details regarding the theory of hyperbolic conservation laws we refer the reader to~\cite{garavello2006traffic,Evans1998}.

\paragraph*{Networks}

A network of hyperbolic conservation laws such as~(\ref{eq:cons})
is defined as a set of $\nlinks$ links $\links=\intrange 1{\nlinks}$,
with junctions $\jns$. Each junction $\jn\in\jns$ is defined
as the union of  two non-empty sets: the set of $\ninc_{\jn}$ incoming links $\Inc\left(\jn\right)=\tuple{\jlink{\jn}1}{\jlink{\jn}{\ninc_{\jn}}}\subset\links$
and the set of $\nout_{\jn}$ outgoing links $\Out\left(\jn\right)=\tuple{\jlink{\jn}{\ninc_{\jn}+1}}{\jlink{\jn}{\ninc_{\jn}+\nout_{\jn}}}\subset\links$.
Each link $\link\in\links$ has an associated upstream junction $\jup{\link}\in\jns$
and downstream junction $\jdown{\link}\in\jns$, and has an associated
spatial domain $\left(0,L_{\link}\right)$ over which the evolution
of the state on link $\link$, $\rho_{\link}\left(t,x\right)$, solves
the Cauchy problem:

\begin{equation}
\begin{cases}
\left(\rho_{\link}\right)_{t}+f\left(\rho_{\link}\right)_{x} & =0\\
\rho_{\link}\left(0,x\right) & =\initstate_{\link}\left(x\right)
\end{cases}\label{eq:cauchy-i}
\end{equation}
where $\initstate_{\link}\in BV\cap L_{\text{loc}}^{1}\left(L_i;\mathbb{R}\right)$
is the initial condition on link $\link$. For simplicity of notation,
this section considers a single junction $\jn\in\jns$ with $\Inc\left(\jn\right)=\left(1,\ldots,\ninc\right)$
and $\Out\left(\jn\right)=\left(\ninc+1,\ldots,\ninc+\nout\right)$.
\begin{rem}
There is redundancy in the labeling of the junctions, if link
$i$ is directly upstream of link $j$, then we have $\jdown{\link}=\jup j$.
See Fig.~\ref{fig:Space-discretization-for}.
\end{rem}
While the dynamics on each link $\rho_{\link}\left(t,x\right)$ is
determined by~(\ref{eq:cauchy-i}), the dynamics at junctions
still needs to be defined.



\begin{defn}
\label{def:Riemann-Problem}Riemann Problem.

A Riemann problem is a Cauchy problem~\eqref{eq:CP} with a piecewise-constant initial datum (called the Riemann datum):

\[
\initstate(x)=\begin{cases}
\rho_{-} & x<0\\
\rho_{+} & x\ge 0
\end{cases}
\]
%with one point of discontinuity, $x=\bar{x}$. Without loss of generality,
%we may take $\bar{x}=0$.
\end{defn}
%It can be shown that the $\rho$ solution generated from the Riemann
%data $\left(\rho_{-};\rho_{+}\right)$ has a constant value along
%lines of constant $\frac{x}{t}$. 
We denote the corresponding self-similar entropy weak solutions by $\ss{\frac{x}{t}}{\rho_{-}}{\rho_{+}}$.





\begin{defn}
Riemann problem at junctions. 

A Riemann problem at $\jn$ is a Cauchy problem corresponding to an initial datum $\tuple{\initstate_{1}}{\initstate_{\ninc+\nout}}\in\mathbb{R}^{\ninc+\nout}$ which is constant on each link $\link.$

%Let all links $\link$ have a constant initial profile $\rho_{\link}\left(0,x\right)=\initstate_{\link}\in\mathbb{R}$
%(called the Riemann data) with incoming links having a spatial domain
%$\left(-\infty,0\right)$ and outgoing links having a spatial domain
%$\left(0,\infty\right)$. The solution of $\rho_{\link}\left(x,t\right)$
%for all links $\link\in\Inc\left(\jn\right)\cup\Out\left(\jn\right)$
%and $t\ge0$ is defined as the solution to the Riemann problem at
%junction $J$ with initial data $\tuple{\initstate_{1}}{\initstate_{\ninc+\nout}}\in\mathbb{R}^{\ninc+\nout}$.
\end{defn}
\begin{defn}
A Riemann solver is a map that assigns a solution to each Riemann initial data. For each junction $\jn$ it is a function

\begin{eqnarray*}
\RS: & \mathbb{R}^{m+n} & \rightarrow\mathbb{R}^{m+n}\\
 & \tuple{\initstate_{1}}{\initstate_{\ninc+\nout}} & \mapsto\RS\tuple{\initstate_{1}}{\initstate_{\ninc+\nout}}=\tuple{\trace{\rho}_{1}}{\trace{\rho}_{\ninc+\nout}}
\end{eqnarray*}
where $\trace{\rho}_{\link}$ provides the trace for link $\link$
at the junction for all time $t\ge0$.

\end{defn}
For a link $i\in\Inc\left(\jn\right)$,
the solution $\rho_{i}\left(t,x\right)$ over its spatial domain
$x<0$ is given by the solution to the following Riemann problem:

\begin{equation}
\begin{cases}
\left(\rho_{\link}\right)_{t}+f\left(\rho_{\link}\right)_{x} & =0\\
\rho_{\link}\left(0,x\right) & =\begin{cases}
\initstate_{\link} & x<0\\
\trace{\rho}_{\link} & x\ge0,
\end{cases}
\end{cases}\label{eq:riemann-problem}
\end{equation}

The Riemann problem for an outgoing link is defined similarly, with
the exception that $\rho_{\link}\left(0,x>0\right)=\initstate_{\link}$
and $\rho_{\link}\left(0,x\le0\right)=\trace{\rho}_{\link}$. 

Fig.~\ref{fig:Solution-of-boundary}
gives a depiction of Riemann solution at the junction.%

\begin{figure}
\begin{centering}
\includegraphics[width=0.5\columnwidth]{presentation/figs-gen/junctions-riemann-rs} 
\par\end{centering}

\caption{Solution of boundary conditions at junction. The boundary conditions
$\tuple{\trace{\rho}_{1}}{\trace{\rho}_{5}}$ are produced by applying
the Riemann solver to the initial conditions, $\tuple{\initstate_{1}}{\initstate_{5}}$.\label{fig:Solution-of-boundary}}
\end{figure}


Note that the following properties for the Riemann Solver holds:
\begin{itemize}
\item All waves produced from the solution to Riemann problems on all links,
generated by the boundary conditions at a junction, must emanate out
from the junction. Moreover, the solution to the Riemann problem
on an incoming link must produce waves with negative speeds, while
the solution on an outgoing link must produce waves with positive
speed. 
\item The sum of all incoming fluxes must equal the sum of all outgoing
fluxes: 
\[
\sum_{i\in\Inc\left(\jn\right)}f\left(\trace{\rho}_{\link}\right)=\sum_{j\in\Out\left(\jn\right)}f\left(\trace{\rho}_{j}\right).
\]
This condition guarantees mass conservation at junctions.
\item The Riemann solver must produce self-similar solutions, i.e. 
\[
\RS\left(\RS\tuple{\initstate_{1}}{\initstate_{\ninc+\nout}}\right)=\RS\tuple{\initstate_{1}}{\initstate_{\ninc+\nout}}=\tuple{\trace{\rho}_{1}}{\trace{\rho}_{\ninc+\nout}}
\]
\end{itemize}

The justification for these conditions can be found in~\cite{garavello2006traffic}.

The above conditions are not always sufficient to guarantee a unique Riemann solver. Additional conditions are added for specific applications to achieve uniqueness, chosen to model physical phenomena at junctions. In Section~\ref{sec:Applications-to-Optimal}, we detail the additional conditions added to the ramp-metering solver which enforce flux maximization along the freeway mainline sections and specify a merging priority model for vehicles entering from the onramps.


\subsection{Godunov Discretization\label{sub:Godunov-Discretization}}

In order to find approximate solutions we use the classical Godunov scheme~\cite{godunov1959}. We use the following notation: $x_{j+\frac{1}{2}}$ are the cell interfaces and   $t^{\tind}=k\Delta t$ the time with $\tind\in\mathbb{N}$ and $\xind\in\mathbb{Z}$. $x_{\xind}$ is the center of the cell, $\Delta x=x_{j+\frac{1}{2}}-x_{j-\frac{1}{2}}$ the cell width, and $\Delta t$ is the time step. 
\paragraph{Godunov scheme for a single link.}

The Godunov scheme is based on the solutions of exact Riemann problems. The main idea of this method is to approximate the initial datum by a piecewise constant function, then the corresponding Riemann problems are solved exactly and a global solution is found by piecing them together. Finally one takes the mean on the cell and proceed by iteration. Given $\rho(t,x),$ the cell average of $\rho$ at time $t^{\tind}$ in the cell $C_{\xind}=]x_{j-\frac{1}{2}},x_{j+\frac{1}{2}}]$ is given by 
\begin{equation}
\discrete{\xind}{\tind}=\dfrac{1}{\Delta x}\int_{\xdis{\xind-\frac{1}{2}}}^{\xdis{\xind+\frac{1}{2}}}\dvar(t^{k},x)dx.\label{eq:godproj}
\end{equation}
Then we proceed as follows:
\begin{enumerate}
	\item We solve the Riemann problem at each cell interface $x_{j+\frac{1}{2}}$ with initial data $(\dvar^{\tind}_{\xind},\dvar^{\tind}_{\xind+1}).$
	\item Compute the cell average at time $t^{\tind +1}$ in each computational cell and obtain $\dvar^{\tind +1}_{\xind}.$ 
\end{enumerate}

We remark that waves in two neighbouring cells do not intersect before $\Delta t$ if the following Courant\textendash{}Friedrichs\textendash{}Lewy (CFL) condition holds, $\lambda^{\max}\le\frac{\Delta x}{\Delta t}$, where $\lambda^{\max}=\underset{a}{\max}|f'\left(a\right)|$ is the maximum wave speed of the Riemann solution at the interfaces.\\
Godunov scheme can be expressed as follows:
\begin{equation}
\discrete{\xind}{\tind+1}=\discrete{\xind}{\tind}-\frac{\Delta t}{\Delta x}(\god(\discrete{\xind}{\tind},\discrete{\xind+1}{\tind})-\god(\discrete{\xind-1}{\tind},\discrete{\xind}{\tind})),\label{eq:godscheme}
\end{equation}
where $g^{G}$ is the Godunov numerical flux given by

\begin{eqnarray*}
\god: & \mathbb{R}\times\mathbb{R} & \rightarrow\mathbb{R}\\
 & \left(\discrete{\xind}{},\discrete{\xind+1}{}\right) & \mapsto\god\left(\discrete{\xind}{},\discrete{\xind+1}{}\right)=f(W_{R}(0;\discrete{\xind}{},\discrete{\xind+1}{})).
\end{eqnarray*}

where $W_{R}$ is as defined in Definition~\ref{def:Riemann-Problem}.


%To approximate a hyperbolic PDE of the form in~(\ref{eq:cons-smooth})
%with a discretization in both space and time, we use Godunov's scheme~\cite{godunov1959}.
%Given an initial condition that is piecewise-constant, Godunov's scheme
%gives a procedure to advance the system one time-step forward. Under
%suitable stability conditions, this process can be repeated \emph{ad
%infinum }by applying the scheme to the new condition generated from
%the previous time-step.

%We define a numerical grid using the following notations 
%\begin{itemize}
%\item $\Delta x$ is the fixed space grid size,
%\item $\Delta t$ is the fixed time grid size, chosen with respect to the Courant\textendash{}Friedrichs\textendash{}Lewy (CFL) condition%
%\footnote{This method can be generalized to non-uniform grid sizes, but is chosen
%to be uniform to simplify notation%
%}, 
%\item $(t^{\tind},\xdis{\xind})=(k\Delta t,\xind\Delta x)$ for $\tind\in\mathbb{N}$
%and $\xind\in\mathbb{Z}$ grid points%
%\footnote{The CFL condition, $\lambda^{\max}\le\frac{\Delta x}{\Delta t}$,
%guarantees that the waves in two neighboring cells do not interact
%at the cell boundaries before $\Delta t$. Above $\lambda^{\max}=\underset{a}{\max}|f'\left(a\right)|$
%is the maximum absolute wave speed. We assume henceforth that this
%condition always holds for the chosen step sizes.%
%}. For a link $\link\in\links$ with space domain $\left(0,\length_{\link}\right)$,
%$\xind$ takes values in $\intrange 0{\length_{\link}^{\Delta}}$.
%\end{itemize}
%For a function $\dvar$ defined at the grid we write $\discrete{\xind}{\tind}=\dvar(t^{\tind},\xdis{\xind})$
%for $\tind,\xind$ varying on a subset of $\mathbb{N}$ and $\mathbb{Z}$.
%The space discretization scheme is depicted for a link $\link\in\links$
%in Fig.~\ref{fig:Space-discretization-for}
\begin{figure}
\begin{centering}
\includegraphics[width=0.6\columnwidth]{figs-gen/dx}
\par\end{centering}

\caption{Space discretization for a link $\link\in\links$. Step size is uniform
$\Delta x$, with discrete value $\discrete{\xind}{\tind}$ representing
the state between $x^{\xind-1}$ and $x^{\xind}$.\label{fig:Space-discretization-for}}


\end{figure}

%Consider a hyperbolic equation together with an initial condition
%as expressed in~(\ref{eq:cauchy}). A solution of the discretized
%problem is constructed by taking a piecewise constant approximation
%of the initial data $\initvar{\dvar}^{\Delta}$ and then constructing
%the solution iteratively from the approximate initial data as in~(\ref{eq:godproj}).
%\begin{equation}
%\discrete{\xind}{\tind}\approxeq\dfrac{1}{\Delta x}\int_{\xdis{\xind-\frac{1}{2}}}^{\xdis{\xind+\frac{1}{2}}}\dvar^{\Delta}(t^{k},x)dx.\label{eq:godproj}
%\end{equation}
%
%
%where $\rho^{\Delta}\left(t,x\right)$ is an exact solution of~(\ref{eq:cauchy})
%with initial condition $\initstate^{\Delta}$.
%
%Since $\rho^{\Delta}$ in~(\ref{eq:godproj}) is an exact solution
%of~(\ref{eq:cons-smooth}) we can use the Gauss-Green formula to
%compute this value. Under the CFL condition, the solutions are locally
%given by the solutions of the Riemann problems and, in particular,
%the flux at $x=\xdis{\xind+\frac{1}{2}}$ for $t\in(t^{k},t^{k+1})$
%is given by 
%\[
%f(\rho(t,\xdis{\xind}))=f(W_{R}(0;\discrete{\xind}{\tind},\discrete{\xind+1}{\tind})).
%\]
%This flux solution is depicted in Fig.~\ref{fig:Self-similar-solution-for}
\begin{figure}
\begin{centering}
\includegraphics[width=0.5\columnwidth]{figs-gen/dx-to-riemann}
\par\end{centering}

\caption{Self-similar solution for Riemann problem with initial data $\left(\discrete{\xind}{\tind},\discrete{\xind+1}{\tind}\right)$.
The self-similar solution at $\frac{x}{t}=0$ for the top diagram
(i.e. $\ss 0{\discrete{\xind}{\tind}}{\discrete{\xind+1}{\tind}}$),
gives the flux solution to the discretized problem in the bottom diagram.\label{fig:Self-similar-solution-for}}
\end{figure}
%.The flux at $x=\xdis{\xind-\frac{1}{2}}$ for $t\in(t^{k},t^{k+1})$
%is given by 
%\[
%f(\rho(t,\xdis{\xind-\frac{1}{2}}))=f(W_{R}(0;\discrete{\xind-1}{\tind},\discrete{\xind}{\tind})).
%\]
%As the flux is constant at $x=0$ for the Riemann problem, we can
%put it out of the integral and, setting $\god(u,v)=f(W_{R}(0;u,v))$,
%the scheme can be written as 
%\begin{equation}
%\discrete{\xind}{\tind+1}=\discrete{\xind}{\tind}-\frac{\Delta t}{\Delta x}(\god(\discrete{\xind}{\tind},\discrete{\xind+1}{\tind})-\god(\discrete{\xind-1}{\tind},\discrete{\xind}{\tind})),\label{eq:godscheme}
%\end{equation}
%where $g^{G}$ is the numerical flux:
%
%\begin{eqnarray*}
%\god: & \mathbb{R}\times\mathbb{R} & \rightarrow\mathbb{R}\\
% & \left(\discrete{\xind}{},\discrete{\xind+1}{}\right) & \mapsto\god\left(\discrete{\xind}{},\discrete{\xind+1}{}\right)=f(W_{R}(0;\discrete{\xind}{},\discrete{\xind+1}{}))
%\end{eqnarray*}



\paragraph{Godunov scheme at junctions.\label{par:Godunov-scheme-at}}

The scheme just discussed applies to the case in which a single cell
is adjacent to another single cell. Yet, at junctions, a cell may
share a boundary with more than one cell. A more general Godunov flux
can be derived for such cases. For incoming links near the junction,
we have: 
\begin{align*}
\discrete{\length_{\link}^{\Delta}}{\tind+1}=\discrete{\length_{\link}^{\Delta}}{\tind}-\dfrac{\Delta t}{\Delta x}(f(\trace{\dvar}_{\length_{\link}^{\Delta}}^{\tind})-\god(\discrete{L_{i}^{\Delta}-1}{\tind},\discrete{L_{i}^{\Delta}}{\tind})), &  & \link\in\left\{ 1,\ldots,\ninc\right\} 
\end{align*}
where $L_i^{\Delta}$ are the number of cells for link $i$ (see Fig.~\ref{fig:Space-discretization-for}) and $\hat{\dvar}_{i}^{\tind}$ is the solution of the Riemann solver
$\RS\tuple{\discrete 1{\tind}}{\discrete{\ninc+\nout}{\tind}}$ for
link $\link$ at the junction. The same can be done for the outgoing
links: 
\begin{align*}
\discrete 1{\tind+1}=\discrete 1{\tind}-\dfrac{\Delta t}{\Delta x}(\god(\discrete 1{\tind},\discrete 2{\tind})-f(\trace{\dvar}_{1}^{\tind})), &  & \link\in\left\{ \ninc+1,\ldots,\ninc+\nout\right\} 
\end{align*}

\begin{rem}
Using the Godunov scheme, each mesh grid at a given $t^{\tind}$ can
be seen as a node for a 1-to-1 junction with one incoming and one
outgoing link. It is therefore more convenient to consider that every
discretized cell is, rather, a link with both an upstream and downstream
junction. Thus, we consider networks in which the state of each link
$\link\in\links$ at a time-step $k\in\intrange 0{\ntime-1}$ is represented
by the single discrete value $\discrete{\link}{\tind}$.
\end{rem}
The previous remark allows us to develop a generalized update step
for all discrete state variables. We first introduce a definition
in order to reduce the cumbersome nature of the preceding notation.
Let the state variables adjacent to a junction $\jn\in\jns$ at a
time-step $\tind\in\intrange 0{\ntime-1}$ be represented as $\juncstate{\jn}{\tind}\defeq\tuple{\discrete{\link_{\jn}^{1}}{\tind}}{\discrete{\link_{\jn}^{\ninc_{\jn}+\nout_{\jn}}}{\tind}}$.
Similarly, we let the solution of a Riemann solver be represented
as $\junctrace{\jn}{}\defeq\RS\left(\juncstate{\jn}{}\right)$. Then,
for a link $\link\in\links$ with upstream and downstream junctions,
$\jup{\link}$ and $\jdown{\link}$, and time-step $\tind\in\left\{ 0,\ldots,\ntime-1\right\} $,
the update step becomes:

\begin{align}
\discrete{\link}{\tind+1} & =\discrete{\link}{\tind}-\dfrac{\Delta t}{\Delta x}\left(f\left(\left(\RS\left(\juncstate{\jdown{\link}}{\tind}\right)\right)_{\link}\right)-f\left(\left(\RS\left(\juncstate{\jup{\link}}{\tind}\right)\right)_{\link}\right)\right)\nonumber \\
 & =\discrete{\link}{\tind}-\dfrac{\Delta t}{\Delta x}\left(f\left(\left(\junctrace{\jdown{\link}}{}\right)_{\link}\right)-f\left(\left(\junctrace{\jup{\link}}{}\right)_{\link}\right)\right)\label{eq:reg-update}
\end{align}
where $\left(s\right)_{i}$ is the $i$th element of the tuple $s$.
This equation is thus a general way of writing the Godunov scheme
in a way which applies everywhere, including at junctions.


\paragraph{Working directly with flux solutions at junctions.\label{par:Composing-the-Riemann}}

The equations can be simplified if we do not explicitly represent
the solution of the Riemann solver, $\junctrace{\jn}{}$, and, instead,
directly calculate the flux solution from the Riemann data. We denote
this direct computation by $\god_{\jn}$, the Godunov flux solution
at a junction:

\begin{eqnarray}
\god_{\jn}: & \mathbb{R}^{\ninc_{\jn}+\nout_{\jn}} & \rightarrow\mathbb{R}^{\ninc_{\jn}+\nout_{\jn}}\nonumber \\
 & \juncstate{\jn}{} & \mapsto f\left(RS\left(\juncstate{\jn}{}\right)\right)=\left(f\left(\trace{\dvar}_{1}\right),\ldots,f\left(\trace{\dvar}_{\ninc+\nout}\right)\right)\label{eq:god-jn}.
\end{eqnarray}
This gives a simplified expressions for the update step:

\begin{equation}
\discrete{\link}{\tind+1}=\discrete{\link}{\tind}-\dfrac{\Delta t}{\Delta x}\left(\left(\god_{\jdown{\link}}\left(\juncstate{\jdown{\link}}{\tind}\right)\right)_{\link}-\left(\god_{\jup{\link}}\left(\juncstate{\jup{\link}}{\tind}\right)\right)_{\link}\right)\label{eq:composed-flux}.
\end{equation}



\paragraph{Full discrete solution method.\label{par:Full-solution-method}}

We assume a discrete scalar hyperbolic network of PDEs with links
$\links$ and junctions $\jns$, and a known discrete state at time-step
$\tind$, $\left(\initdiscrete_{\link}^{\tind}:\link\in\links\right)$.
The solution method for advancing the discrete system forward one
time-step is given in Algorithm~(\ref{algo:rs-alg}), or alternatively
Algorithm~(\ref{algo:god-alg}).

\begin{algorithm}[H]
\caption{\texttt{Riemann solver update procedure}}


\lstinputlisting[basicstyle={\ttfamily\footnotesize},breaklines=true,label={algo:rs-alg},mathescape=true]{rs-alg}
\end{algorithm}


Algorithm~\ref{algo:rs-alg} takes as input the state at a time-step
$\tind$ for all links $\left(\discrete{\link}{\tind}:\link\in\links\right)$
and returns the state advanced by one time-step $\left(\discrete{\link}{\tind+1}:\link\in\links\right)$.
The algorithm first iterates over all junctions $\jn$, calculating
all the boundary conditions, $\junctrace{\jn}{\tind}$. Then, the
algorithm iterates over all links $\link\in\links$ to compute the
updated state $\discrete{\link}{\tind+1}$ using the previously computed
boundary conditions, as in~\ref{eq:reg-update}.

\begin{algorithm}[h]
\caption{\texttt{Godunov junction flux update procedure}}


\lstinputlisting[basicstyle={\ttfamily\footnotesize},breaklines=true,label={algo:god-alg},mathescape=true]{god-alg}
\end{algorithm}


Algorithm~\ref{algo:god-alg} is similar to Algorithm~\ref{algo:rs-alg},
except that the boundary conditions $\junctrace{\jn}{\tind}$ are
not explicitly computed, but rather the Godunov flux solution is used
to update the state, as in~\ref{eq:composed-flux}. Algorithm~\ref{algo:god-alg}
is more suitable if a Godunov flux solution is derived for solving
junctions, while Algorithm~\ref{algo:rs-alg} is more suitable if
one uses a Riemann solver at junctions.
