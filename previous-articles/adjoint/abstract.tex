%!TEX root = restart.tex
\begin{abstract}
The adjoint method provides a
computationally efficient means of  calculating the gradient for applications
in constrained optimization. In this  article, we consider a network of scalar
conservation laws with general topology, whose behavior is modified by a set
of control  parameters  in order to minimize a given objective function. After
discretizing the corresponding partial differential equation models  via the
Godunov scheme, we detail the computation of the gradient of the discretized
system with respect to the control parameters and  show that the complexity of
its computation scales linearly with the  number of discrete state variables
for networks of small vertex degree. The method is applied to solve the
problem of coordinated ramp metering on freeway networks. Numerical
simulations on the I15 freeway in California demonstrate an improvement in
performance and running time compared to existing methods, and a robustness to noise in the initial data and boundary conditions demonstrated in the context of model predictive control.
\end{abstract}
