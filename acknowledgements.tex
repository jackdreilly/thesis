% -*- root: thesis.tex -*-
I would first like to thank Professor Alexandre Bayen for inviting me into his lab five years ago and for working closely with me during the entirety of that time. From him I learned much about rigor, thoroughness, persuasiveness, leadership and professionalism: traits which I have found to be the most valuable assets gained during my PhD. I am forever grateful for his patience and guidance.

I would also like to thank Professor Roberto Horowitz, Dr. Gabriel Gomes, Dr. Ajith Muralidharan, and the other researchers in Professor Horowitz's lab. My work has benefited greatly from the different approaches to research and traffic in which I have participated. In particular, their work on linear formulations of freeway optimal control problems~\cite{gomes2006optimal,Muralidharana} greatly improved the practical nature of the theory developed within this thesis. Additionally, Gabriel's suggested extensions of my work to new problems such as network sensitivity analysis and model calibration have illuminated a broader applicability of the underlying research.

I give my gratitude to Professor Scott Moura, Professor Alexander Skabardonis, Professor Eli Yablonovitch, and Professor Steven Glaser for their guidance during my qualifying examination. Their feedback was valuable in directing the remainder of my research.


One of the most productive and enjoyable periods during my PhD was spent at the INRIA research center in Nice, France during Fall 2012, working with Dr. Paola Goatin and her student Maria Laura Delle Monache. I gained much insight into traffic models in general and how to develop a combined ordinary differential equation and partial differential equation model of freeway traffic. Their knowledge and methodology (e.g. mandatory coffee breaks) have permanently affected me.

I would also like to thank Raphael Marinier and Mihai Stroe of Google Road Traffic for their mentorship during my internship in Zurich, Switzerland in Summer 2013. Raphael's expertise in data analysis and focus on concrete results are skills I admire and attempt to emulate. I very much enjoyed the plethora of interesting traffic-related problems they have and their open-mindedness towards their solutions.

I thank again Professor Steven Glaser and thank Professor Raja Sengupta for welcoming me into the Civil Systems program mere days before the start of my Masters. Their novel approach to civil engineering is what reinvigorated my interest in the area, and I still regard my rash decision to switch into the Systems program as the best decision I have made.

I have had the pleasure to collaborate with many different students during my time in Berkeley. Specifically, I was lucky enough to work closely with Samitha Samaranayake over the last five years on many different projects and in many different venues (Sutardja Dai, La Val's, Nice, Tahoe, Gilman...). Additionally, I benefited greatly from the intelligence and philosophical musings of S\'{e}bastien Martin while working on freeway traffic security.

A distinguishing feature of my PhD was the ability to implement my research within production systems. I would like to thank Branko Kerkez and Mario Magliocco for their infinite patience with a greenhorn as I worked on the iShake project. I would also like to thank Joe Butler and many others, including Dimitris Triantafyllos, at CCIT and PATH for their dedication to supporting research in a professional environment and professionalism in a research environment.

To my dearest friends, Devon, Jack, Matty, Zack, Heidi, Timothy, Katelyn, thank you for embodying what I find good in this world.

Finally and foremost, I give my thanks and love to my mother Joanne, father Jim, and my brothers and best friends Jimmy and Christopher.